\section*{A. CLI コマンドリファレンス}
本研究で開発した CLI ツール \texttt{auto-iot} の主要コマンドおよびその仕様を以下に示す。

\subsection*{A.1 init-bot-map}
SwitchBot API と通信を行い、アカウントに紐付いている物理デバイスの一覧を取得して、マッピング定義ファイルの雛形 (\texttt{switchbot.map.json}) を生成する。

\begin{verbatim}
$ auto-iot init-bot-map [options]
\end{verbatim}

\noindent \textbf{Options:}
\begin{description}
  \item[\texttt{-d, --output-dir <dir>}] \hfill \\
  出力先のディレクトリを指定する。ファイル名は \texttt{switchbot.map.json} で固定される。(デフォルト: \texttt{.})
  
  \item[\texttt{-c, --conflict <mode>}] \hfill \\
  同名のファイルが既に存在する場合の挙動を指定する。
  \begin{itemize}
    \item \texttt{overwrite}: 上書きする (デフォルト)。
    \item \texttt{add}: 既存のファイルに新しいデバイスを追記・マージする。
    \item \texttt{skip}: 生成をスキップする。
  \end{itemize}
\end{description}

\subsection*{A.2 generate-doc (Analyzer)}
マニュアル (PDF/テキスト) を解析し、中間表現 (IR) および OpenAPI 仕様書を生成する。
\texttt{--prompt} オプション等を使用することで、生成時の追加指示を与えることが可能である。

\begin{verbatim}
$ auto-iot generate-doc [options]
\end{verbatim}

\noindent\textbf{Options:}
\begin{description}
  \item[\texttt{--doc <file>}] \hfill \\
  \textbf{[必須]} 解析対象のマニュアルファイルのパス (PDF またはテキスト)。
  
  \item[\texttt{--map <file>}] \hfill \\
  \textbf{[必須]} \texttt{init-bot-map} で作成したデバイスマップのパス。
  
  \item[\texttt{--base <file>}] \hfill \\
  既存の API 仕様書がある場合、それをベースとして読み込む (省略可)。
  
  \item[\texttt{-p, --prompt <text>}] \hfill \\
  追加の指示プロンプトを文字列で指定する (省略可)。
  
  \item[\texttt{--prompt-file <file>}] \hfill \\
  追加の指示プロンプトを記述したテキストファイルのパス (省略可)。
  
  \item[\texttt{-o, --output-dir <dir>}] \hfill \\
  生成物 (\texttt{openapi.json}, \texttt{switchbot\_flow.json} 等) を保存するディレクトリ。
\end{description}

\subsection*{A.3 generate-api (Synthesizer)}
OpenAPI 仕様書を入力とし、OpenAPI Generator を用いて API サーバのコードを生成する。
\texttt{--with-markers} オプションを指定することで、ロジック注入用マーカーを含んだコードが生成される。

\begin{verbatim}
$ auto-iot generate-api [options]
\end{verbatim}

\noindent \textbf{Options:}
\begin{description}
  \item[\texttt{-i, --input <path>}] \hfill \\
  \textbf{[必須]} 入力となる OpenAPI 仕様書 (YAML) のパス。
  
  \item[\texttt{-o, --output <path>}] \hfill \\
  \textbf{[必須]} コードを出力するディレクトリ。
  
  \item[\texttt{--with-markers}] \hfill \\
  マーカー注入用テンプレートを使用して生成する (デフォルト: false)。本システムでは必須。
  
  \item[\texttt{-g, --generator <name>}] \hfill \\
  使用するジェネレータ名 (例: \texttt{python-flask})。
  
  \item[\texttt{-c, --config <path>}] \hfill \\
  OpenAPI Generator の設定ファイルパス。
\end{description}

\subsection*{A.4 implement (Injector)}
操作フロー定義 (\texttt{switchbot\_flow.json}) を読み込み、各 \texttt{operationId} に対応するマーカー領域に制御ロジックを注入する。

\begin{verbatim}
$ auto-iot implement [options]
\end{verbatim}

\noindent \textbf{Options:}
\begin{description}
  \item[\texttt{--flow <file>}] \hfill \\
  \textbf{[必須]} 操作フロー定義ファイルのパス。
  
  \item[\texttt{--root, --project-root <dir>}] \hfill \\
  \path{openapi_server/} が含まれるプロジェクトルート (デフォルト: \texttt{.})。
  
  \item[\texttt{--strict}] \hfill \\
  注入対象のマーカーが見つからない場合にエラーとして扱う (デフォルト: false)。
  
  \item[\texttt{--dry-run}] \hfill \\
  書き込みを行わず、注入内容をプレビューする (デフォルト: false)。
\end{description}

