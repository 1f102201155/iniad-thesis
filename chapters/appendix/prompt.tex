\section*{B. プロンプト定義}
本システムのマニュアル解析フェーズ (Analyzer) において、LLM (GPT-4o) に入力されるプロンプトの構成を示す。
実装コード (\path{src/app/services/doc.ts}) より、システムプロンプト、ユーザー入力構成、およびチャンク要約時の指示を抜粋したものである。

\subsection*{B.1 System Prompt}
LLM の役割定義と、出力すべき中間表現 (IR) の JSON スキーマを厳密に定義している。ハルシネーション(幻覚)を防ぐため、OpenAPI を直接出力させず、独自の中間形式に限定している点が特徴である。

\begin{tcblisting}{
  breakable,
  enhanced,
  skin=enhancedlast,
  colback=white,
  colframe=black!50!white,
  listing only,
  title=System Prompt,
}
You are a senior engineer who extracts ONLY control operations from a home-appliance manual and outputs a minimal IR (Intermediate Representation) as a single JSON object

Output requirements:
- Output a SINGLE JSON object (no code fences).
- DO NOT output OpenAPI here; output IR only: { info, operations, flow }.

## IR schema (mandatory)
{
  "info": { "title"?: string, "version"?: string },
  "operations": [
    { "path": string, "method": "post"|"put"|"delete"|"get", "operationId": string,
      "summary"?: string, "description"?: string,
      "parameters"?: any[], "requestBody"?: any }
  ],
  "flow": {
    "version"?: string,
    "flows": [
      { "operationId": string,
        "steps": [ { "type": "press"|"on"|"off", "targetLabel": string, "holdMs"?: number, "delayAfterMs"?: number } ]
      }
    ]
  }
}

## Strict rules
- Exclude any status polling or state queries.
- steps[*].type must be one of: press / on / off.
- steps[*].targetLabel must EXACTLY match one of VALID_LABELS provided (do NOT translate or paraphrase).
- If device mode is pressMode, use only "press"; if switchMode, use only "on"/"off".
- Only include button operations that exist in switchbot.map.json; any label not in the map is strictly forbidden.
\end{tcblisting}

\subsection*{B.2 User Prompt (Initial Context)}
解析対象となるデバイス情報(マップファイル)と、ユーザーからの追加指示、および既存の API 仕様書(存在する場合)をコンテキストとして与えるプロンプトである。ここで \texttt{VALID\_LABELS} を提示し、ラベルの表記揺れを抑制している。

\begin{tcblisting}{ 
  breakable,
  enhanced,
  skin=enhancedlast,
  colback=white,
  colframe=black!50!white,
  listing only,
  title=User Prompt (Intro)
}
Existing base spec (optional; integrate if helpful):
{baseSpec}

User request (optional):
{promptText}

switchbot.map.json:
{botMapRaw}

VALID_LABELS (use EXACTLY one of these; do NOT translate or paraphrase):
- 電源ボタン (mode=pressMode)
- スタート (mode=pressMode)
...

I will send the manual in several chunks. Do NOT output the final JSON until the last instruction.
\end{tcblisting}

\subsection*{B.3 Chunk Summarization Prompt}
長大なマニュアルを分割(チャンク化)して入力する際、各チャンクに対して送信される要約指示である。ここでは最終的な JSON は出力させず、操作手順の抽出に専念させている。

\begin{tcblisting}{ 
  breakable,
  enhanced,
  skin=enhancedlast,
  colback=white,
  colframe=black!50!white,
  listing only,
  title=Summarization Instruction ,
}
Manual chunk {i}/{total}
---
{chunkText}

From this chunk, summarize in 3-6 lines focusing on:
- Which labeled buttons (by label) to press and in what order to achieve the intended operations.
- Ignore any status polling, conditional branching, or sensor-dependent logic.
Do NOT output the final JSON yet.
\end{tcblisting}

\subsection*{B.4 Final Instruction}
全チャンクの要約が完了した後、最終的な JSON オブジェクトの生成を指示するプロンプトである。これまでの文脈を統合し、IR スキーマに適合した出力を強制する。

\begin{tcblisting}{ 
  breakable,
  enhanced,
  skin=enhancedlast,
  colback=white,
  colframe=black!50!white,
  listing only,
  title=Final Instruction,
}
Considering all of the above, output MINIMAL IR as SINGLE JSON object (no code fences).
Format: { "info"?: {...}, "operations": [...], "flow": {...} }
- Do NOT output OpenAPI here.
- In "operations", list only the necessary endpoints (path, method=post is preferred, operationId, summary).
- In "flow", create a 1:1 mapping with "operations": for each operationId, provide steps with {type, targetLabel}.
- targetLabel must be an EXACT match from VALID_LABELS. Do NOT invent new labels.
- If you are not certain which label to use, OMIT that operation/step rather than guessing.
\end{tcblisting}

