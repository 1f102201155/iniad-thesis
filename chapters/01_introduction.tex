\chapter{はじめに}

\section{研究背景}
近年,IoT (Internet of Things) 機器の普及に伴い,スマートホーム市場が急速に拡大している。
総務省の調査によれば,世界のIoTデバイス数は年々増加傾向にあり,
家庭内においても照明,エアコン,掃除機,スマートロックなど多種多様な機器がネットワークに接続されるようになった\cite{SoumuIoT2023}。

これに伴い,家電の操作インターフェースは従来の物理的なリモコンやスイッチから,
スマートフォンアプリやスマートスピーカーを介したソフトウェア制御へと移行しつつある。
ユーザはインターネットを経由して外出先から家電の状態を確認したり,
複数の機器を連携させた自動制御(ホームオートメーション)を行ったりすることで,
生活の利便性とエネルギー効率を飛躍的に向上させることが可能となった。

しかし,市場の拡大は同時にシステム環境の複雑化も招いている。
現在普及している多くの IoT 家電は,メーカーごとに独自のクラウドサービスや通信仕様(プロトコル)を採用しており,
\textbf{互いに独立し,連携が困難な状態にある}。
家庭内には異なるメーカーの機器が混在し,これらを統一的に管理・制御するためには,
各機器が提供する API (Application Programming Interface) を適切に利用し,
システム間で連携させる技術が不可欠となっている。

\subsection{本研究が対象とするIoTシステム構成}
図\ref{fig:iot-basic-structure}に,スマートホームを含む一般的なIoTシステムの基本構成を示す。
左側上段は \textbf{ネットワークへ直接接続可能なデバイス}(スマートフォンやPC,セルラー/Wi-Fi内蔵機器)であり,
これらはクラウドと直接通信してサービスを利用する。
左側下段は \textbf{ネットワークへ直接接続しない(できない)デバイス}で,BLEやZigbee,赤外線などローカル無線で
\textbf{ゲートウェイ}(ハブ)に接続し,ゲートウェイがセルラーやブロードバンド経由でクラウドへ中継する。
中央の\textbf{ネットワーク}(インターネット/モバイル網)を介して,
右側の\textbf{サーバ}(クラウド)では認証・認可,レート制限,ログ収集,状態管理などの運用機能が実装される。

本研究では,この一般構成を前提に,\emph{クラウドAPI層とデバイス/ハブ層の間}でのインタフェース記述(OpenAPI)と
APIサーバ実装の自動生成に着目する。

\begin{figure}[tb]
  \centering
  \includegraphics[width=.95\linewidth]{figures/iot-basic-structure.jpg}
  \caption{IoTの基本構成(直結デバイス/非直結デバイス+ゲートウェイ/ネットワーク/サーバ)。
  図は Seeds4biz「【徹底解説】IoTを構成する技術要素|身体の機能に置き換えて考えてみよう」 の解説ページより引用\cite{SeedsIoTComponents}。}
  \label{fig:iot-basic-structure}
\end{figure}

しかし,IoT システムの構築と運用には,依然として高い技術的障壁が存在する。
これらを統合して動作させるためには,物理層からアプリケーション層までを繋ぎ合わせるフルスタックな実装スキルが要求される。
特に,ハードウェアに依存した通信仕様やデータ形式(バイナリデータや独自プロトコル)の理解は,
一般的な Web アプリケーション開発者にとって学習コストが高く,
これが新規サービス開発やシステム統合の参入障壁となっている。

\section{現状の課題}
IoT 機器の相互運用性(Interoperability)を確保するために,
ECHONET Lite や Matter といった標準規格の策定が進められている\cite{ECHONETSpec,MatterSpec}。
これらの規格に準拠した機器であれば,共通のインターフェースを通じて操作が可能である。

しかし現実には,標準規格に対応していないレガシーな家電や,
コスト制約により独自の簡易な通信仕様を採用した製品が市場には数多く存在する\cite{SpaceCore2024}。
また,標準規格に対応している機器であっても,メーカー独自の機能(拡張仕様)については
標準プロトコルでカバーしきれない場合が多い。

こうした「規格外」の機器をシステムに統合するためには,
開発者が各製品の取扱説明書や技術マニュアル(仕様書)を読み込み,
操作に必要なコマンド体系やパラメータを一つ一つ理解し,
それらをプログラムコード(ドライバ)として実装する必要がある\cite{SmartDev2024}。
この「マニュアル読解から実装まで」のプロセスは,
対象機器が増えるたびに繰り返す必要があり,極めて労働集約的である。
また,記述の曖昧な自然言語のマニュアルから正確な仕様を読み取る作業は,
ドメイン知識を持たない開発者にとっては困難であり,実装ミスやバグの温床となりやすい。

\section{本研究のアプローチ}
本研究では,大規模言語モデル(Large Language Model; LLM)の高い自然言語処理能力と
コード生成能力を活用することで,IoT 機器統合における「専門知識の壁」を取り除くことを目指す。
具体的には,IoT 機器の取扱説明書(PDF)を入力とするだけで,
その機器を操作するための Web API(OpenAPI 仕様)と,
実際に動作するサーバ実装コードを全自動で生成するシステムを提案する。

提案システムは,マニュアルから操作手順や通信仕様を抽出し,
IoT 制御に特化した中間表現(Intermediate Representation; IR)を経て,
標準的な REST API 定義である OpenAPI Specification (OAS) を生成する。
さらに,OpenAPI Generator を用いてサーバサイドのコード(Flask)を生成することで,
開発者はマニュアルを用意するだけで,即座に機器を Web API 経由で操作可能となる。

\section{本研究の目的と成果}
本研究の主な目的と貢献は以下の 2 点である。

\begin{enumerate}
  \item \textbf{SwitchBot Bot を介した物理操作によるレガシー家電の IoT 化手法の確立}:
  API を持たない物理デバイスに対し,SwitchBot Bot による物理的なボタン操作を抽象化・統合するシステムを構築した。
  これにより,既存の家電を買い替えることなく,あらゆる機器を即座にスマートホームの一部として制御可能であることを実証した。

  \item \textbf{LLM を用いた操作仕様の構造化と自動生成パイプラインの構築}:
  取扱説明書(非構造データ)から操作仕様を抽出し,OpenAPI 形式および制御コードへ変換する一貫した自動化パイプラインを開発した。
  本手法により,組み込み等の専門知識を持たない開発者でも容易に API を生成可能とした。
  さらに,本手法が IoT 開発のプロトタイピングにおける工数削減と構文的信頼性の確保に有効であることを明らかにした。
\end{enumerate}

\section{論文の構成}
本論文の構成は以下の通りである。
第2章では,IoT の相互運用性に関する既存技術および LLM を用いた仕様生成に関する関連研究について述べる。
第3章では,提案手法のコンセプトである,マニュアルから API 仕様を自動生成するアプローチの概要について説明する。
第4章では,提案手法を実現するためのシステム設計について述べ,データフローや各コンポーネントの役割を定義する。
第5章では,設計に基づいた実装の詳細について,使用したライブラリや環境を含めて述べる。
第6章では,複数の実際の家電マニュアルを用いた評価実験を行い,生成成功率,操作フローの整合性,および生成の一貫性について検証する。
第7章では,実験結果に基づいた考察を行い,本手法の有効性と限界,および今後の課題について議論する。
最後に第8章で本研究を総括し,結論を述べる。
