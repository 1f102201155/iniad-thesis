\section{考察}
\label{sec:discussion}

本節では,実験結果に基づき提案手法の有効性について考察する。
また,本研究の成果を基盤とした将来的な発展の可能性についても述べる。

\subsection{中間表現 (IR) を介したハイブリッド生成の有用性}
本研究の最大の特徴は,LLM に最終的なプログラムコードを直接生成させるのではなく,
API の仕様と操作フローを表す中間表現 (IR) の生成に役割を限定した点にある。
実験結果において,GPT-4o を用いた場合の生成成功率が 93.3\% に達したことは,
このアプローチが構文エラーの抑制に極めて有効であることを示唆している。
非決定的な LLM の出力を決定論的なテンプレートエンジンで補完する設計は,
生成 AI をシステム開発に適用する上での有効な解の一つといえる。

\subsection{マニュアルから実機操作までの一貫した自動化}
本システムにより,従来は開発者がマニュアルを読み込み,仕様を理解し,
コードを記述する必要があった一連の工程を,PDF の入力のみで完結させることが可能となった。
第6章の実験では,生成された API サーバがエラーなく起動し,
定義されたフローに従ってドライバを呼び出せることが確認された。
これにより,高度な専門知識を持たないユーザであっても,
家電を操作するための初期システムを数分程度で構築することが可能となり,
IoT サービス開発におけるプロトタイピングのコストを大幅に削減できることが示された。

\subsection{今後の展望}
本研究により確立された自動生成パイプラインは,以下のような機能拡張を行うことで,
より高度な IoT システムへと発展できる可能性を有している。

\subsubsection*{Bot 機能の拡張による操作の多様化}
前述した物理操作の制約を克服するために,
Bot のカスタムモード(Customize Mode)を活用した機能拡張が挙げられる。
これにより,「長押し」や「連続押し」といった動作が可能となる。
具体的には,マニュアル解析時に「操作の長さ(Duration)」や「待機時間(Interval)」といったパラメータを抽出し,
IR のスキーマをこれに対応できるように拡張することで,
より多様な家電操作を記述・実行可能なシステムへの発展が期待できる。

\subsubsection*{センサ連携による閉ループ制御への拡張}
SwitchBot 開閉センサーや温湿度計,あるいは電力計付きプラグなどをシステムに統合し,
操作の結果を外部センサで確認する「閉ループ制御」への拡張も有望である。
LLM がマニュアルから「操作」だけでなく「期待される状態変化(例: 室温が上がる,消費電力が増える)」も抽出し,
それを検証ロジックとしてコードに埋め込むことができれば,
より信頼性の高い自律的な IoT システムが実現できると考えられる。

\subsubsection*{実運用に向けたセキュリティ機構の統合}
本システムで生成される API を家庭内だけでなく,外部ネットワークからも安全に利用可能にするため,
OAuth 2.0 などの標準的な認可プロトコルや,通信経路の暗号化(HTTPS)を自動構成する機能の追加も考えられる。
また,生成されたコードに対する脆弱性スキャンをパイプラインに組み込み,
セキュリティリスクを事前に排除する「DevSecOps」的なアプローチの導入によって,
実運用に耐えうるシステムへと昇華させることが可能である。
