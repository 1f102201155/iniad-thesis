\section{評価指標}
\label{sec:metrics}

本実験では、提案システムの有効性と信頼性を定量的・多角的に評価するため、以下の4つの指標を定義する。

\subsection{生成成功率}
生成されたコードが構文的に正しく、システムとして有効に機能するかを評価する指標である。
具体的には、OpenAPI 仕様書(YAML/JSON)および Flask サーバーコードの生成プロセスがエラーなく完了し、
生成されたサーバーが正常に起動可能な状態にある場合を「成功」と定義する。

総試行回数を $N_{total}$、そのうち成功した試行回数を $N_{success}$ とし、生成成功率 $R_{success}$ を次式で算出する。

\begin{equation}
  R_{success} = \frac{N_{success}}{N_{total}} \times 100 \quad [\%]
\end{equation}

\subsection{操作フローの正確性}
マニュアルに記載された操作手順が、生成された API のロジックとして正しく抽出されているかを評価する。
本研究では、マニュアルの内容を基に人手で作成した「正解操作フロー(Ground Truth)」を用意し、
生成された操作フローと比較することで、以下の3つの値を算出する。

\begin{itemize}
  \item \textbf{True Positive ($TP$)}: 正解データに存在し、かつ生成結果にも正しく含まれている操作ステップ数。
  \item \textbf{False Positive ($FP$)}: 正解データには存在しないが、生成結果に誤って含まれている操作ステップ数(過剰抽出)。
  \item \textbf{False Negative ($FN$)}: 正解データには存在するが、生成結果に含まれていない操作ステップ数(抽出漏れ)。
\end{itemize}

これらの値を用い、抽出の正確性を測る指標として、適合率(Precision)、再現率(Recall)、およびそれらの調和平均である F1 スコア(F1-score)を次式で定義する。

\begin{equation}
  Precision = \frac{TP}{TP + FP}
\end{equation}

\begin{equation}
  Recall = \frac{TP}{TP + FN}
\end{equation}

\begin{equation}
  F1\text{-}score = \frac{2 \times Precision \times Recall}{Precision + Recall}
\end{equation}

適合率は「生成された操作のうち正しいものの割合」、再現率は「抽出されるべき操作のうち実際に抽出された割合」を表し、F1 スコアは両者のバランスを示す総合的な指標である。

\subsection{生成の一貫性}
LLM は確率的な生成モデルであるため、同一の入力に対しても出力結果が異なる場合がある。
IoT ドライバのようなシステム開発においては、出力の安定性が重要となる。
そこで、同一のマニュアルを入力として複数回生成を行い、その出力結果の類似度を「一貫性」として評価する。

本実験では、各試行で生成された操作フロー構造(JSONオブジェクト)を比較し、試行間での一致率(Flow Consistency Rate)を算出する。
$N$ 回の試行におけるフローの一致度を $C_{flow}$ とし、完全に同一のフローが出力された割合、もしくはフロー間の構造的類似度(0.0 〜 1.0)の平均値として定義する。

\subsection{生成時間}
システムの実用性を評価するため、マニュアル(PDF)の入力から、最終的な API サーバーコードが出力されるまでの処理時間(秒)を計測する。
これには、PDF の解析、中間表現(IR)の生成、OpenAPI 仕様の生成、およびコード生成の全工程が含まれる。

\subsection{トークン効率}
LLM を用いたシステムにおいて、消費トークン数は運用コストに直結する重要な要素である。
本実験では、1回のドライバ生成プロセスにおいて消費された総トークン数(入力プロンプトと出力補完の合計)を計測し、モデル間のコスト効率を比較する。
トークン数が少ないほど、API 利用コストが低く、経済的なモデルであると判断できる。
