\section{評価項目}

本研究では,提案手法によって自動生成される API 仕様およびコードの品質を多面的に検証するため、
以下の 6 つの指標を用いて評価を行う。これらの指標は,
仕様生成の正確性、手順整合性、コードの実行可能性、再現性、
および効率性を総合的に把握することを目的としている。

\begin{enumerate}
    \item \textbf{生成成功率}  
          自動生成処理がエラーなく最後まで完了し,生成された OpenAPI 仕様が構文的に正しいかを評価する指標である。
          本指標は,仕様生成パイプラインの安定性や堅牢性を確認することを目的とする。

    \item \textbf{操作フロー整合性}  
          生成された API 仕様および flow.json が,マニュアルに記載された操作手順とどの程度一致しているかを測る指標である。
          操作の過不足やボタンの対応関係,ステップ構造などの正確性を評価する。

    \item \textbf{アプリ起動性}  
          自動生成された Flask サーバコードが追加修正なしで起動可能かを評価する指標である。
          これはコード生成テンプレートおよび LLM が生成した operationId 一貫性の検証に対応する。

    \item \textbf{生成時間}  
          API 仕様生成(generate-doc)からコード生成(generate-api)の完了までに要する処理時間を評価する。
          モデルごとの処理速度や実運用における応答性の観点を把握する目的がある。

    \item \textbf{生成一貫性}  
          同一の入力(マニュアル PDF)に対して複数回生成を行った際に,
          LLM が生成する flow.json の構造がどの程度安定して再現されるかを評価する指標である。
          LLM 出力のゆらぎを測るために導入した。

    \item \textbf{トークン効率}  
          LLM API 呼び出しにおけるプロンプトトークン,応答トークンの消費量を比較し,
          モデルごとの効率性や運用コストへの影響を評価する指標である。
\end{enumerate}

各指標の算出結果を後述の Table~\ref{tab:eval-summary} に示す。
