\section{実験環境}
本節では,評価に用いたハードウェア構成,ソフトウェア構成,
LLM モデルと利用期間,ネットワーク環境,
および評価対象データについて述べる。

%---------- ハードウェア構成 ---------------
\subsection{ハードウェア構成}
評価実験には,表 ~\ref{tab:eval-hw} に示す構成の PC を用いた。
LLM 呼び出し,OpenAPI 仕様生成,コード生成はすべてこの環境で実行した。

\begin{table}[tb]
  \centering
  \caption{評価に用いたハードウェア構成}
  \label{tab:eval-hw}
  \begin{tabular}{|l|l|}
    \hline
    項目   & 内容 \\ \hline
    CPU   & 12th Gen Intel(R) Core(TM) i7-12650H \\ \hline
    GPU   & NVIDIA GeForce MX550 \\ \hline
    メモリ & 32\,GB \\ \hline
    OS    & Windows 11 \\ \hline
    実行環境 & PowerShell 7.5.4 \\ \hline
  \end{tabular}
\end{table}

%---------- ソフトウェア構成 ---------------
\subsection{ソフトウェア構成}
主要な開発環境およびミドルウェアのバージョンを
表 ~\ref{tab:eval-sw} に示す。
本研究で開発した CLI ツール(auto-iot)は Node.js 上で動作し,
LLM の呼び出しや OpenAPI Generator の起動を内部で行う。

\begin{table}[tb]
  \centering
  \caption{評価に使用した主なソフトウェア環境}
  \label{tab:eval-sw}
  \begin{tabular}{|l|l|}
    \hline
    項目 & バージョン・内容 \\ \hline
    Python & 3.12.10 \\ \hline
    Node.js & 23.5.0 \\ \hline
    パッケージマネージャ & npm(npx 経由で CLI 実行) \\ \hline
    OpenAPI Generator & 7.17.0\footnotemark[1] \\ \hline
    Auto-IoT CLI(本研究ツール) & version 0.1.0 \\ \hline
  \end{tabular}
\end{table}

\footnotetext[1]{以下のコマンドにより Java 版 OpenAPI Generator のバージョンを確認した。
\texttt{npx @openapitools/openapi-generator-cli version} の実行時に
バイナリが自動的にダウンロードされ,7.17.0 が選択されたことを確認している。}

%---------- LLMモデルおよび利用期間 ---------------
\subsection{LLM モデルおよび利用期間}
本研究では OpenAI 社の Chat Completions API を用いて LLM 呼び出しを行った。
評価対象としたモデルは以下の 2 つである。

\begin{itemize}
  \item GPT-4o
  \item GPT-4o-mini
\end{itemize}

API 呼び出しは 2025 年 10 月 27 日から 11 月 5 日までの期間に実施し,
この期間に提供されていた最新版のモデルを使用した。

%---------- 評価対象データと実行条件 ---------------
\subsection{評価対象データと実行条件}
評価対象として,10 種類の家電製品の取扱説明書(PDF ファイル)を収集した。
各マニュアルに対し,以下の手順で評価データを準備した。

また,本研究では各マニュアルに対して \textbf{3 回ずつ} API 生成処理(generate-doc)を実行した。
そのため,評価に用いた生成結果は \textbf{合計 30 件} となる。

\begin{enumerate}
  \item 各家電のマニュアル PDF を 1 件ずつ用意する。
  \item CLI サブコマンド \texttt{init-bot-map} により
        \texttt{switchbot.map.json} の雛形を生成する。
  \item appliance, button, mode などの項目を手動で補完し,
        マニュアルに記載されたボタン・操作と整合するよう編集する。
\end{enumerate}
