\section{評価方針}
\label{sec:policy}

本研究の目的は,高度なドメイン知識を持たない開発者であっても,
マニュアルを用意するだけで IoT 機器の操作 API を容易に構築可能にすることである。
前節で述べた通り,本評価では物理デバイスを用いた実機検証は行わず,
生成されたソフトウェア成果物の品質と,生成プロセスの実用性に焦点を当てる。

具体的には,以下の 3 つの観点から定量的な評価を行う。

\subsection{生成物の機能的妥当性}
生成された成果物が,システムとして破綻なく機能するかを評価する。
LLM は時に構文的に誤った JSON やコードを出力する場合がある(ハルシネーション)。
本システムにおいては,中間表現から OpenAPI 仕様書を経てサーバーコードに至るパイプラインが
エラーなく完走し,最終的に Flask サーバーが正常に起動・待機状態になれるかを
「生成成功」の最低条件として定義する。

\subsection{マニュアル記述との整合性}
生成された API が,入力としたマニュアルの内容を正しく反映しているかを評価する。
サーバーが起動しても,操作コマンドが不足していたり,
誤ったパラメータが設定されていては実用性がない。
そのため,マニュアルに記載されている操作フロー(手順)が,
過不足なく API のロジックとして抽出されているかを,
正解データ(Ground Truth)との比較により検証する。

\subsection{コスト効率とモデル特性}
実用的なシステム運用において,API 利用コスト(トークン量)や生成時間は重要な要素である。
一般に,高性能モデルは高コスト・低速であり,軽量モデルは低コスト・高速であるとされる。
本実験では,GPT-4o と GPT-4o-mini の 2 つのモデルで生成を行い,
消費トークン数や所要時間を計測することで,
IoT ドライバ生成タスクにおいてどちらのモデルがコスト対効果(Cost-Performance)に優れているかを明らかにする。
