\chapter{提案手法}
\label{chap:proposed_method}

本章では,家電マニュアルから IoT 操作用 Web API および制御コードを自動生成するための提案手法について述べる。
本研究の核となるのは,大規模言語モデル(LLM)の柔軟な読解能力と,従来のアルゴリズムによる堅牢なコード生成を組み合わせた「ハイブリッド生成アプローチ」である。

\section{システム概要}
提案システムの全体処理フローを図\ref{fig:system_flow}に示す。
本システムは,入力として「家電の取扱説明書(PDF/テキスト)」および「デバイスマップ(物理デバイスとラベルの対応表)」を受け取り,
以下の 3 段階の処理を経て,最終的に動作可能な API サーバのソースコードを出力する。

\begin{enumerate}
  \item \textbf{マニュアル解析と中間表現(IR)の生成}:
  LLM を用いてマニュアルから操作に関する情報のみを抽出し,本システム独自の軽量な JSON フォーマット(中間表現)に構造化する。
  \item \textbf{仕様書と制御フローの合成}:
  生成された IR を基に,決定論的なアルゴリズムを用いて標準的な OpenAPI 仕様書(OAS)と,デバイス操作手順書(Flow 定義)を自動合成する。
  \item \textbf{コード生成と実装注入}:
  OpenAPI Generator を用いてサーバのスタブコード(Flask)を生成し,その内部に Flow 定義に基づいた制御ロジックを注入する。
\end{enumerate}

この構成により,LLM 特有の「幻覚(Hallucination)」や「構文エラー」のリスクを最小限に抑えつつ,
多様なフォーマットで記述されたマニュアルへの対応を実現している。

\begin{figure}[tb]
  \begin{center}
    \includegraphics[width=0.5\textwidth]{figures/application_overview.png}
  \end{center}
  \caption{}
  \label{fig:system_flow}
\end{figure}

\section{中間表現 (Intermediate Representation; IR) の導入}
\label{sec:ir_design}

既存の多くの手法では,LLM に直接 OpenAPI 仕様(YAML/JSON)を出力させようとする。
しかし,OpenAPI 仕様は非常に冗長であり,かつ厳密なスキーマ構造(ネストされたオブジェクトや参照定義)を持つため,
LLM が一貫して正しい構文を出力することは困難である。
特に,トークン長制約の中で巨大な仕様書を生成させると,JSON の閉じ括弧の欠落や,必須フィールドの省略といったエラーが多発する。

さらに,本研究における予備検討として,LLM に直接 OpenAPI 仕様(YAML)を生成させるアプローチを試みたところ,
マニュアルの自然言語記述に含まれる「:(コロン)」が適切にエスケープされず,
パースエラーが多発するという実用上の課題も確認された。

そこで本手法では,LLM の出力を API 構築に必要な\textbf{「最小限の情報」}に限定するアプローチをとる。
これを\textbf{中間表現(IR)}と定義する。
IR は以下の 3 つの要素のみを持つシンプルな JSON オブジェクトである。

\begin{itemize}
  \item \textbf{info}: API のタイトル。
  \item \textbf{operations}: API として公開すべき操作のリスト(エンドポイントパス,HTTP メソッド,操作 ID)。
  \item \textbf{flow}: 各操作 ID に対応する具体的なデバイス制御手順(ボタン操作の順序)。
\end{itemize}

パラメータ定義やレスポンス定義,エラーハンドリングといった「定型的な」情報は IR には含めず,
後段のプログラムで自動補完する設計とすることで,LLM の負荷を下げ,生成の安定性を向上させている。

\section{ハイブリッド生成プロセス}

\subsection{Phase 1: LLM による抽出 (Extraction)}
このフェーズでは,自然言語で記述されたマニュアルから IR を生成する。
ここでの LLM の役割は,新たな操作手順を作り出す「創造」ではなく,
マニュアルから必要な情報を抜き出す\textbf{「抽出」}と,
自然言語を機械可読な形式へ変換する\textbf{「構造化(マッピング)」}に限定される。

具体的には,マニュアルをテキストチャンクに分割して LLM に入力し,
「どのボタンを,どういう順序で押せば,どのような操作が実現できるか」という情報を抽出させる。
この際,以下の制約をプロンプトで与えることで,出力の品質を担保する。

\begin{itemize}
  \item \textbf{操作の限定}: 電源 ON/OFF やモード変更など,物理ボタンで操作可能な機能のみを抽出対象とし,状態確認(Status Polling)はSwitchBot Botでは実現できないため対象外とする。
  \item \textbf{ラベルの厳密一致}: 操作対象となるボタン名は,後述するデバイスマップに定義されたラベル(Valid Labels)と完全に一致させることを強制する。未知のラベルや,LLM が想像で作ったラベルの使用は禁止する。
\end{itemize}

\subsection{Phase 2: 決定論的合成 (Deterministic Synthesis)}
Phase 1 で生成された IR は,TypeScript で実装されたジェネレータによって処理され,以下の 2 つの成果物に変換される。
この変換プロセスはルールベースのアルゴリズムであり,LLM は介入しないため,常に構文的に正しい出力が保証される。

\begin{enumerate}
  \item \textbf{OpenAPI Specification (openapi.json/yaml)}:
  IR の \texttt{operations} リストに基づき,パスやメソッドを定義する。
  この際,正常系(200 OK)および異常系(401 Unauthorized, 503 Service Unavailable 等)のレスポンススキーマは,
  システム側で用意したテンプレートを全エンドポイントに一律で適用する。
  これにより,手動で記述するには手間のかかる厳密な API 定義を瞬時に生成できる。

  \item \textbf{操作フロー定義 (switchbot\_flow.json)}:
  IR の \texttt{flow} 部分を切り出し,アプリケーションが実行時に参照するための設定ファイルとして保存する。
  ここでは,API の \texttt{operationId} と,実際の物理ボタン操作(\texttt{press}, \texttt{on}, \texttt{off})のマッピングが定義される。
\end{enumerate}

\subsection{Phase 3: コード生成 (Code Generation)}
生成された OpenAPI 仕様を入力として,OSS ツールである OpenAPI Generator を実行し,
Python (Flask) のサーバコードを生成する。
本研究では,生成されたコントローラコード(スタブ)に対し,
Phase 2 で生成したフロー定義を読み込んで実行する汎用的なドライバロジックを注入する。
これにより,開発者は一切のコーディングを行うことなく,マニュアルの内容を反映した IoT API サーバを手に入れることができる。

\section{デバイスマップによる物理抽象化}
第\ref{sec:switchbot_background}節で述べた通り,SwitchBot Bot と Cloud API を利用することで,
通信機能を持たないレガシー家電であっても,物理ボタンへの後付け設置によって API 経由での制御が可能となる。
本研究では,この「物理操作の API 化」という利点を活かすことで,
マニュアルから単に仕様書を作成するだけでなく,実際に物理デバイスを操作可能なシステムを構築する。

家電の種類やメーカーによって,操作に必要な SwitchBot デバイスの構成は異なる(例:物理ボタンを押す「ボット」,赤外線を飛ばす「ハブ」など)。
本手法では,これらの物理デバイス構成を \texttt{switchbot.map.json} という定義ファイルに抽象化して入力する。

しかし,マニュアルに記述されている「ボタン名(ラベル)」と,
実際の API 制御に必要な「デバイス ID」は直接対応していない。
そこで本手法では,両者を仲介する抽象化レイヤとして \textbf{「デバイスマップ」} を導入する。

\subsection{デバイスマップの定義構造}
デバイスマップ (\texttt{switchbot.map.json}) は,
家電の「機能的なラベル」と「物理的なデバイス情報」の対応関係を定義した JSON ファイルである。
このマップにより,以下の 2 点の抽象化を実現している。

\begin{enumerate}
  \item \textbf{ID の隠蔽とラベル解決}:
  LLM は「電源ボタン」「スタート」といった意味的なラベルのみを扱って操作フローを生成する。
  システムは実行時にこのマップを参照し,ラベルに対応する物理デバイス ID(例: \texttt{E16A82C65393})へと変換する。
  これにより,LLM の出力は特定のハードウェア環境に依存しない汎用的なものとなる。

  \item \textbf{動作モードの制約}:
  各デバイスに対して,第\ref{sec:switchbot_background}節で説明した Bot の動作モード
  (\texttt{pressMode} / \texttt{switchMode}) をマップ上で定義する。
  システムはこの定義に基づき,LLM が生成した操作タイプ(\texttt{press} なのか \texttt{on/off} なのか)が
  物理的に実行可能であるかを検証(バリデーション)する。
\end{enumerate}
