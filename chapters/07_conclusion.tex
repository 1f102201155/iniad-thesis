\chapter{結論}
\label{chap:conclusion}

\section{まとめ}
本研究では,多種多様な家電製品を統一的な Web API を通じて操作可能にすることを目指し,
大規模言語モデル (LLM) と OpenAPI Generator を組み合わせた API 自動生成システム「Auto-IoT」を提案・実装した。

本論文を通して得られた知見と成果は,以下の 2 点に集約される。

第一に,\textbf{中間表現 (IR) を介したハイブリッド生成アプローチの有効性}である。
LLM の役割を「意味抽出」に限定し,コード生成を決定論的なテンプレートエンジンに委ねることで,
LLM 単独生成で課題となる構文エラーやハルシネーションを効果的に抑制できることを示した。
評価実験における 93.3\% という高い生成成功率は,このアプローチが実用的な信頼性を担保する上で極めて有効であることを裏付けている。

第二に,\textbf{マニュアル入力から実機操作までの一貫した自動化の実現}である。
提案システムは,PDF の解析からサーバコードの生成,そして物理デバイス(SwitchBot)の制御ロジック注入までを,
人手による修正なしに完結させるパイプラインを確立した。
これにより,高度な専門知識を持たない開発者であっても,マニュアルを用意するだけで即座に IoT システムを構築可能となり,
プロトタイピングの大幅な迅速化と省力化に寄与するという本研究の目的は達成された。

\section{今後の展望}
本研究の成果を発展させるための展望として,
Bot のカスタムモード活用による「長押し」等の複雑な操作への対応や,
外部センサと連携して動作確認を行う「閉ループ制御」の導入による信頼性の向上が挙げられる。
また,実運用を見据えた認証・認可メカニズムやセキュリティ機構の自動統合を進めることで,
より汎用的かつ安全な IoT 構築基盤としての発展が期待できる。
