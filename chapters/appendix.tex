\chapter*{付録}
\addcontentsline{toc}{chapter}{付録}
\label{chap:appendix}

% ==========================================
% A. CLIツールの使い方
% ==========================================
\section*{A. CLI コマンドリファレンス}
本研究で開発した CLI ツール \texttt{auto-iot} の主要コマンドおよびその仕様を以下に示す。

\subsection*{A.1 init-bot-map}
SwitchBot API と通信を行い、アカウントに紐付いている物理デバイスの一覧を取得して、マッピング定義ファイルの雛形 (\texttt{switchbot.map.json}) を生成する。

\textbf{使用法:}
\begin{verbatim}
$ auto-iot init-bot-map [options]
\end{verbatim}

\textbf{オプション:}
\begin{itemize}
  \item \texttt{-d, --output-dir <dir>}: 出力先のディレクトリを指定する。ファイル名は \texttt{switchbot.map.json} で固定される (デフォルト: \texttt{.} )。
  \item \texttt{-c, --conflict <mode>}: 同名のファイルが既に存在する場合の挙動を指定する。
  \begin{itemize}
    \item \texttt{overwrite}: 上書きする (デフォルト)。
    \item \texttt{add}: 既存のファイルに新しいデバイスを追記する。
    \item \texttt{skip}: 生成をスキップする。
  \end{itemize}
  \item \texttt{-h, --help}: ヘルプを表示する。
\end{itemize}

\subsection*{A.2 generate-doc (Analyzer)}
マニュアル (PDF/テキスト) を解析し、中間表現 (IR) および OpenAPI 仕様書を生成する。
本コマンドの実行には \texttt{OPENAI\_API\_KEY} が環境変数として必要である。

\textbf{使用法:}
\begin{verbatim}
$ auto-iot generate-doc [options]
\end{verbatim}

\textbf{オプション:}
\begin{itemize}
  \item \texttt{--doc <file>}: 解析対象のマニュアルファイルのパス (PDF またはテキスト)。
  \item \texttt{--map <file>}: \texttt{init-bot-map} で作成したデバイスマップのパス。
  \item \texttt{--base <file>}: 既存の API 仕様書がある場合、それをベースとして読み込む (省略可)。
  \item \texttt{-o, --output-dir <dir>}: 生成物 (\texttt{openapi.json}, \texttt{openapi.yaml}, \texttt{switchbot\_flow.json}) を保存するディレクトリ。
  \item \texttt{-h, --help}: display help for command
\end{itemize}

\subsection*{A.3 implement}
% ここに implement コマンドの使用例と、ロジック注入の挙動について記述する

% ==========================================
% B. LLMへのプロンプト
% ==========================================
\section*{B. プロンプト定義}
マニュアル解析フェーズにおいて LLM (GPT-4o) に与えた指示の構成を以下に示す。

\subsection*{B.1 System Prompt}
% ここに System Prompt の実物(または主要な制約部分)を枠付きで掲載する

% ==========================================
% C. 生成されるデータの例
% ==========================================
\section*{C. 生成データの詳細例}
システムが出力する中間ファイルおよび最終成果物の構造例を示す。

\subsection*{C.1 中間表現 (IR) の生成例}
% ここに ir.json の実例(JSONデータ)を掲載する

\subsection*{C.2 注入後のサーバコード例}
% ここに default_controller.py の実例(Pythonコード、注入前後がわかるもの)を掲載する

% ==========================================
% D. 主要なソースコード
% ==========================================
\section*{D. システムの主要実装コード}
本システムの動作原理を支える重要なコンポーネントの実装を抜粋して示す。

\subsection*{D.1 SwitchBot ドライバ (\texttt{switchbot\_driver.py})}
% ここに Python ドライバのコード(署名生成や通信部分)を掲載する

\subsection*{D.2 ロジック注入処理 (TypeScript 抜粋)}
% ここに implement.ts 内の「正規表現でマーカーを探して置換するロジック」を掲載する
