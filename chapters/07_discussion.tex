\chapter{考察および今後の課題}
\label{chap:discussion}

本章では,第6章の評価実験の結果に基づき,提案手法の有効性と限界について考察する。
また,明らかになった課題を踏まえ,今後の研究の発展性について展望を述べる。

\section{有効性のまとめ}
\subsection{中間表現 (IR) を介したハイブリッド生成の有用性}
本研究の最大の特徴は,LLM に最終的なプログラムコードを直接生成させるのではなく,
API の仕様と操作フローを表す中間表現 (IR) の生成に役割を限定した点にある。
実験結果において,GPT-4o を用いた場合の生成成功率が 93.3\% に達したことは,
このアプローチが構文エラーの抑制に極めて有効であることを示唆している。
非決定的な LLM の出力を決定論的なテンプレートエンジンで補完する設計は,
生成AIをシステム開発に適用する上での有効な解の一つといえる。

\subsection{マニュアルから実機操作までの一貫した自動化}
本システムにより,従来は開発者がマニュアルを読み込み,仕様を理解し,
コードを記述する必要があった一連の工程を,PDF の入力のみで完結させることが可能となった。
第6章の実験では,生成された API サーバがエラーなく起動し,
定義されたフローに従ってドライバを呼び出せることが確認された。
これにより,高度な専門知識を持たないユーザであっても,
家電を操作するための初期システムを数分程度で構築することが可能となり,
IoT サービス開発におけるプロトタイピングのコストを大幅に削減できることが示された。

\section{課題と限界}
\subsection{物理操作の複雑性に対する制約}
本研究では,プロトタイピングの迅速化とパイプラインの検証を優先するため,
Bot の動作を単純な「単一ボタンの押下(Single Press)」のみに限定した。
しかし,実際の家電操作には「長押し(例:3秒以上)」や「同時押し」,
あるいは「ダイヤルを回す」といった複雑な物理操作を要求するものが数多く存在する。
現状のシステムではこれらの操作手順を含むマニュアルを正確に再現できないため,
適用可能な家電製品の種類が単純なボタンスイッチを持つものに限られる点が限界として挙げられる。

\subsection{モデル特性によるトレードオフ}
実験結果より,高性能モデルである GPT-4o は形式遵守能力(JSON 構文の正確さ)に優れる一方,
マニュアルからの情報抽出能力(F1 スコア)においては,軽量モデルである GPT-4o-mini に劣るという結果が得られた。
これは,GPT-4o が文脈を過剰に解釈し,存在しない操作を補完(ハルシネーション)したり,
操作名の表記を揺らがせたりする傾向が強いためと考えられる。
用途に応じて,抽出タスクには軽量モデル,構造化タスクには高性能モデルを使い分けるといった
モデル選定の最適化が課題となる。

\subsection{開ループ制御による信頼性の限界}
本システムは,第3章で述べた通り,デバイスの状態確認を行わない「開ループ制御」を採用している。
そのため,操作コマンドが正常に送信されたとしても,
「物理的にボタンが押せなかった(Bot が剥がれた)」「赤外線が届かなかった」といった物理的な失敗を検知することができない。
また,「現在の状態が OFF の場合のみ ON にする」といった条件付き操作も実現できないため,
ストーブや鍵の施錠など,高い信頼性が求められる家電への適用にはリスクが伴う。

\subsection{セキュリティに関する課題}
自動生成された API サーバは,基本的なトークン認証やアクセス制御の仕組みを備えているものの,
商用レベルの IoT システムに求められる高度なセキュリティ要件(通信の暗号化強度の担保,
DoS 攻撃への耐性,ファームウェア改ざん検知など)まではカバーしていない。
特に,家庭内ネットワークに接続される IoT 機器はセキュリティリスクの標的となりやすいため,
生成されたコードを実運用する際には,別途セキュリティ層(リバースプロキシや VPN 等)を
組み合わせるなどの対策が必要となる。

\section{今後の展望}
\subsection{Bot機能の拡張による複雑な操作への対応}
前述した物理操作の制約を克服するために,
Bot のカスタムモード(Customize Mode)を活用した機能拡張が挙げられる。
これにより,「長押し」や「連続押し」といった動作が可能となる。
具体的には,マニュアル解析時に「操作の長さ(Duration)」や「待機時間(Interval)」といったパラメータを抽出し,
IR のスキーマをこれに対応できるように拡張することで,
より多様な家電操作を記述・実行可能なシステムへと発展させる必要がある。

\subsection{センサ連携による閉ループ制御への拡張}
前述した信頼性の限界を克服するため,
SwitchBot 開閉センサーや温湿度計,あるいは電力計付きプラグなどをシステムに統合し,
操作の結果を外部センサで確認する「閉ループ制御」への拡張が期待される。
LLM がマニュアルから「操作」だけでなく「期待される状態変化(例: 室温が上がる,消費電力が増える)」も抽出し,
それを検証ロジックとしてコードに埋め込むことができれば,
より信頼性の高い自律的な IoT システムが実現できるだろう。

\subsection{実運用に向けたセキュリティ機構の統合}
本システムで生成される API を家庭内だけでなく,外部ネットワークからも安全に利用可能にするため,
OAuth 2.0 などの標準的な認可プロトコルや,通信経路の暗号化(HTTPS)を自動構成する機能の追加が求められる。
また,生成されたコードに対する脆弱性スキャンをパイプラインに組み込み,
セキュリティリスクを事前に排除する「DevSecOps」的なアプローチの導入も検討すべきである。
