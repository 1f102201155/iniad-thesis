\chapter{関連研究}
\label{chap:related_works}
本章では,本研究の背景となる既存のアプローチと関連技術について整理する。
まず,(1) 通信機能を持たない機器をネットワークに接続する「レトロフィット」の手法とその課題について述べ,
(2) IoT システムにおける相互運用性の現状と,非スマート機器が取り残されている問題について指摘する。
その解決策の技術要素として,(3) LLM による API 仕様生成技術,
および (4) OpenAPI 仕様を用いたコード生成技術について概説する。
最後に,本研究で採用する (5) 物理ボタン操作デバイスの制御技術について述べ,
これらを踏まえた本研究の立ち位置を明確にする。

% =========================================================
% 2.1 非スマート家電のIoT化手法とその課題 
% =========================================================
\section{非スマート家電のIoT化手法とその課題}

通信機能を持たない非スマート家電を IoT システムに統合するためには,
外部から何らかの物理的・電気的な介入を行い,事後的にネットワーク接続能力を付与する必要がある。
本節では,このようなアプローチを「IoT レトロフィット」と定義し,
その主要な手法と適用範囲,および課題について整理する。

\subsection{非スマート家電におけるレトロフィットの定義}
一般に「レトロフィット (Retrofit)」とは,旧式の装置や建造物に新しい技術や機器を組み込み,
機能を更新することを指す。
本研究における IoT レトロフィットとは,製造年が古い機器に限らず,
現在販売されているものの通信機能を持たない家電製品 (Non-smart Device) に対し,
後付けのデバイスを介してネットワーク制御機能を追加することと定義する。
これにより,高価なスマート家電への買い替えを行うことなく,
既存の資産を活用して安価にスマートホーム環境を構築することが可能となる。

\subsection{レトロフィット手法の分類と適用範囲}
レトロフィットを実現する手法は,対象機器への介入方法によって主に以下の 3 つに分類できる。

\begin{enumerate}
  \item \textbf{電源制御アプローチ (Smart Plug)} \\
  AC 電源の供給をリレーによって ON/OFF することで制御する手法である。
  代表的な製品として \textbf{TP-Link Kasa} シリーズ\cite{TPLinkKasa}や \textbf{Meross Smart Plug} などがある。
  
  この手法は,物理スイッチが機械的に ON の状態で固定できる
  単純な機器(扇風機,電気スタンド,電気ケトルなど)に対しては極めて有効である。
  しかし,マイコン制御を採用している現代的な家電(洗濯機,炊飯器,空気清浄機など)は,
  通電後に待機状態となり,別途「スタートボタン」の押下を必要とするため,
  電源制御だけでは動作を開始できないという致命的な制限がある。

  \item \textbf{信号エミュレーションアプローチ (IR Blaster)} \\
  家電付属のリモコンが発する赤外線 (IR) 信号を学習し,
  ゲートウェイデバイスから再送信することで制御する手法である。
  \textbf{Nature Remo}\cite{NatureRemo} や \textbf{SwitchBot Hub}\cite{SwitchBotHub} が代表的である。
  
  この手法は対象機器の物理的な改造を必要とせず,
  テレビやエアコン,照明器具などの AV・空調機器に対して広く普及している。
  しかし,赤外線受光部を持たない調理家電,家事家電,あるいは給湯器の操作パネルなどには適用できず,
  物理ボタンしか持たない機器は制御対象外となる。


  \item \textbf{物理操作アプローチ (Physical Actuation)} \\
  小型のアクチュエータを用いて,家電の物理ボタンを機械的に押下する手法である。
  代表的な製品として \textbf{SwitchBot Bot} や \textbf{Adaprox Fingerbot}\cite{Fingerbot} が挙げられる。
  
  この手法の最大の利点は,その汎用性にある。
  通信ポートや赤外線受光部を持たない機器であっても,物理的なボタンさえ存在すれば,
  それを外部から押下することで制御が可能となる。
  したがって,前述の 2 手法では対応できない洗濯機,給湯器,インターホン,オートロック解錠ボタンなど,
  多種多様なレガシー機器を IoT 化する唯一の非破壊的かつ後付け可能な解となる。
\end{enumerate}

\subsection{物理操作アプローチの優位性と実装課題}
以上の比較から,家庭内に存在する多種多様な Non-smart デバイスを包括的に IoT システムへ統合するためには,
第 3 の「物理操作アプローチ」が最も適していると言える。
機器の分解や配線加工を伴わずに設置できる点も,大きな利点である。

しかし,この物理操作アプローチには,他の手法にはない特有の課題が存在する。
それは「操作ロジックの実装コスト」である。
スマートプラグであれば「ON」のみ,スマートリモコンであれば「学習した信号の送信」のみで機能が完結するのに対し,
物理操作デバイスの場合,対象機器のユーザーインターフェース (UI) に合わせて,
「どのボタンを」「どのような順序で」「どれくらいの間隔で」操作するかというシーケンス(ドライバ)を
人間が設計・実装しなければならない。

例えば,「洗濯機の標準コース」を実行するためには,
「電源ボタンを押す」→「起動待機(2秒)」→「スタートボタンを押す」といった一連の手順が必要となる。
対象機器が増えるたびに,開発者がマニュアルを読み解き,このような制御ロジックを個別にコーディングすることは,
労働集約的であり,拡張性の観点で大きなボトルネックとなっている。
本研究は,この物理操作特有の「実装の壁」を,LLM による自動生成技術によって取り除くことを目的とする。

% =========================================================
% 2.2 IoT における相互運用性と標準化の限界
% =========================================================
\section{IoT における相互運用性と標準化の限界}
前節で述べた物理的なレトロフィットのアプローチに対し,
既存の IoT 業界では「相互運用性 (Interoperability)」をどう確保しようとしているか,
そしてなぜそれが非スマート家電の救済において不十分であるかを整理する。

\subsection{標準化プロトコルによるアプローチ}
IoT 機器の断片化(Fragmentation)を解消するため,
通信プロトコルやデータモデルを標準化する取り組みが長年行われてきた。
国内においては ECHONET Lite,国際的には Matter などの
標準規格が策定されている\cite{ECHONETSpec,MatterSpec}。
これらの規格に準拠した機器同士であれば,事前の統合作業なしに相互操作が可能となる。

\subsection{標準規格に取り残される Non-smart デバイス}
しかし,これらの標準化アプローチは,あくまで「最初から通信機能を備えた機器」を前提としている。
一方,市場や家庭内には,通信機能を持たない「非スマート家電 (Non-smart Device)」が依然として多数存在する。
これには,製造から年数が経過した旧来の製品だけでなく,
コスト制約により通信モジュールを搭載していない現行製品も含まれる\cite{SpaceCore2024}。

標準規格の普及は進んでいるものの,これら Non-smart デバイスはその枠組みから取り残されており,
デジタルの世界と統合するためには,前述したような物理的なレトロフィットと,
それを制御するための個別ドライバ実装が不可欠となっている。
本研究は,この「標準化ではカバーできない領域」を埋めるためのアプローチである。

\subsection{IoT プラットフォームによる統合と「手動実装」の壁}
プロトコルが異なる機器を統合する別のアプローチとして,
Home Assistant や Eclipse SmartHome といった IoT プラットフォームが広く利用されている。
これらは「Binding」や「Integration」と呼ばれるドライバモジュールを介すことで,
HTTP,MQTT,Bluetooth などの異なるプロトコルを抽象化し,統一的な操作を可能にする。

しかし,これらの統合モジュールの開発は,コミュニティの開発者による手動実装に強く依存している。
新しい機器が発売されるたびに,専門知識を持つ開発者がマニュアルや通信パケットを解析し,
個別にコードを記述する必要があるため,
新製品への追従性やメンテナンスコストが大きなボトルネックとなっている\cite{SmartDev2024}。
この「手動実装」の壁こそが,本研究が自動化によって解決しようとする課題である。

\subsection{Web of Things (WoT) と記述モデルの限界}
W3C が推進する Web of Things (WoT) は,
Thing Description (TD) と呼ばれる JSON-LD 形式のメタデータを用いて機器のインターフェースを記述し,
Web 技術標準での統合を目指している。
WoT のアプローチは,機器の実装(プロトコル)とインターフェース記述を分離する点で,
本研究の OpenAPI によるアプローチと親和性が高い。

しかし,既存の(特に非スマートな)機器に対して適切な Thing Description を付与するためには,
やはり人間が仕様を理解し,記述を作成する必要がある。
本研究は,この「仕様記述の作成」という最も専門性を要する工程を,
LLM によるマニュアル読解によって自動化する試みとして位置づけられる。

% ---------------------------------------------------------
\section{LLMによる API 仕様生成}

前節で述べた「仕様記述コスト」の問題に対し,
近年では LLM を用いて API 仕様(OpenAPI)を自動生成する研究が活発に行われている。
本節ではオンラインドキュメント,コードなど異なる入力から OpenAPI を生成する手法を概説する。

\subsection{OASBuilder: Online API Documentation からの仕様生成}
OASBuilder\cite{OASBuilder2025} は,オンラインで公開されている API ドキュメントを入力として,
対応する OpenAPI 仕様書を自動生成することを目的とした手法である。
同一著者らによる先行研究 SpeCrawler\cite{SpeCrawler2023} も API ドキュメントを対象に
OpenAPI 仕様を生成する試みであったが,抽出結果の構造化や検証工程が限定的であり,
大規模な仕様では抜け漏れが生じやすいという課題があった。
OASBuilder はこれらの課題を踏まえ,より体系化された中間表現(IR)および検証フェーズを導入することで,
精度と信頼性を向上させている。

また,多くの商用 API ドキュメントは人間の閲覧を前提としており,HTML 構造や記述形式が統一されていないため,
LLM を直接適用すると項目の抜け漏れや不整合が生じやすい。
これに対して OASBuilder は,
まず API ドキュメントから API 名称,パラメータ,レスポンス要素などを段階的に抽出し,
整理された中間表現(Intermediate Representation; IR)として統合する。
その後,IR を基に OpenAPI 仕様へ変換し,さらに生成結果を検証・修正する複数段階のパイプラインを採用している。

このように,長大で階層性の高い仕様情報を一度に生成せず,
\textbf{段階的抽出(decomposition)→構造化(IR)→仕様生成→検証}という逐次的な設計を導入することで,
LLM が不完全な情報から誤った JSON 構造を生成してしまう問題を緩和している。
特に IR の導入は,仕様生成を安定化させる上で重要な役割を果たし,
後続の整合性チェックやエラー修正も容易にする。
この点は,本研究において家電マニュアルとデバイスマップから IR を形成し,
OpenAPI Generator によるコード生成につなげるという構成とも共通しており,
OASBuilder は本研究のアプローチに近い先行研究といえる。

\subsection{LRASGen: ソースコードからの OpenAPI 生成}
LRASGen\cite{LRASGen2025} は,部分的なソースコードから RESTful API の振る舞いを推定し,
対応する OpenAPI 仕様を自動生成することを目的とした手法である。
既存の多くの API 仕様生成手法は,API ドキュメントやコメントを前提としており,
コード断片のみからエンドポイントやパラメータ,レスポンス構造を復元することは困難である。
LRASGenはこの課題に対し,ソースコードからエンドポイントメソッド,エンドポイントパラメータ,
パラメータ制約,エンドポイントレスポンスという四つのエンティティを段階的に同定し,
その結果に基づいて OpenAPI 仕様を生成するパイプラインを提案している。

LRASGen はコードベースのサービスに対して有効であり,
既存コードから API 仕様を後付けで構築する。
一方で,本研究が対象とする IoT 家電では公開ソースコードが提供されない場合が多く,
仕様生成はマニュアル記述や動作仕様など非構造的な情報から行う必要がある。
この点で LRASGen は入力ソースの性質が大きく異なり,
本研究のように説明文・機能一覧・デバイスマップから中間表現(IR)を構築し,
OpenAPI 仕様へ変換するアプローチとは方向性が異なる。
しかし,LRASGen が採用する「段階的抽出と LLM による仕様補完」という構成は,
構造化されていない入力から仕様を推定する際の一般的な設計指針として参考になる。

% ---------------------------------------------------------
\section{OpenAPI仕様からのコード生成}
巨大な OpenAPI 仕様を扱う研究として,LLM によるコード生成の自動化が検討されている。  
本研究では openapi-generator を用いてコード生成を行うため,LLM でコードまで生成する手法とは異なる立場を取る。


\subsection{LLMによる大規模 OpenAPI のコード生成}
大規模な OpenAPI 仕様からコード生成を行う研究として,
Pejcz らによる研究\cite{LargeOpenAPI_CodeGen2023}がある。
この研究では,コンテキスト長を超える規模の OpenAPI 仕様を
そのまま LLM に入力するのではなく,仕様を分割しつつ
プロンプトをオーケストレーションすることで,
REST API 実装コードを段階的に生成する手法を提案している。
本研究とは異なり,仕様生成からコード生成までを LLM が一貫して担う点に特徴があり,
OpenAPI Generator を用いて決定的にコード生成を行う本研究とは
役割分担の設計が異なる。

% ---------------------------------------------------------
\section{物理ボタン操作デバイスの制御技術}
\label{sec:switchbot_background}

本研究では,第2.1節で述べたレトロフィット手法のうち,
最も汎用性が高い「物理ボタン操作」を実現するために,SwitchBot 社のデバイスを採用する。
本節では,本システムが制御対象とするハードウェア特性と API 仕様について述べる。

\subsection{SwitchBot Bot の概要}
SwitchBot Bot(以下,Bot)\cite{SwitchBotBot}は,
家電の物理スイッチやボタン付近に貼り付けて使用する小型のロボットアームデバイスである(図\ref{fig:switchbot_bot_image})。
Bluetooth Low Energy (BLE) 通信によりスマホアプリやハブから制御され,
アームを動かして物理的にボタンを押下することで,
通信機能を持たない既存の家電(照明,炊飯器,給湯器など)を IoT 化することができる。

Bot には以下の 3 つの動作モードが存在する。
\begin{itemize}
  \item \textbf{プレスモード (Press Mode)}:
  「押して戻る」動作を行うモード。
  状態を持たないプッシュボタン(例: 炊飯器のスタートボタン)の操作に適している。
  \item \textbf{スイッチモード (Switch Mode)}:
  専用パーツを用いて「押す(ON)」と「引き上げる(OFF)」を行うモード。
  物理的な状態を持つロッカースイッチ(例: 壁の照明スイッチ)の操作に適している。
  \item \textbf{カスタムモード (Customize Mode)}:
    ON および OFF の操作コマンドに対して,
    それぞれ任意の動作シーケンス(アクションセット)を定義可能なモードである。
    1 つのアクションは複数の「ステップ」から構成され,
    各ステップには「押下時間」,「待機間隔」,「繰り返し回数」を指定することができる。
    これにより,長押しや連打といった複雑な物理操作パターンを実現する。
\end{itemize}

\begin{figure}[b]
  \centering
  \includegraphics[width=0.6\linewidth]{figures/washing_machine_sample_img.jpg}
  \caption{洗濯機への SwitchBot Bot を設置している様子}
  \label{fig:switchbot_bot_image}
\end{figure}

\subsection{SwitchBot Cloud API}
SwitchBot 社は,同社製品をインターネット経由で制御するための Web API (SwitchBot Cloud API) を公開している\cite{SwitchBotCloudAPI}。
開発者は HTTP リクエスト(GET/POST)を送信することで,
デバイスの状態取得やコマンド送信を行うことができる。

Bot に対する制御コマンドは「操作」に特化しており,
「指定した ID の Bot を押す」という単純な命令を実行する。
ただし,Bot は物理的にボタンを押すことはできるが,
その結果として「家電が実際に動作したか(例: 炊飯が始まったか)」を検知するセンサは持たないため,
API からの制御は基本的に一方的なコマンド送信となる特性がある。

% ---------------------------------------------------------
\section{本研究との位置づけ}

以上の既存研究を踏まえると,以下の特徴が本研究の独自性につながる。

\begin{itemize}
  \item 家電マニュアルやデバイスマップから IR を生成し,OpenAPI 仕様を自動生成する点
  \item LLM は仕様生成(IR 形成)に限定し,コード生成は openapi-generator を利用するハイブリッド構成である点
  \item IoT デバイス操作に特化した API を自動生成し,実際の家電操作ドライバに統合する点
\end{itemize}

これらの点から,本研究は既存の仕様生成研究や IoT 自動化研究とは異なる独自の貢献を持つ。
