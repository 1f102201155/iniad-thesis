\section{実装アプローチの概要}
\label{sec:impl_overview}

本システムは,Node.js (TypeScript) で実装された CLI ツール \texttt{auto-iot} として構築されている。
本ツールは以下の 4 つのサブコマンドを提供しており,これらを順次実行することで API の生成を行う。

\begin{itemize}
    \item \texttt{init-bot-map}: SwitchBot API からデバイス一覧を取得し,物理デバイスと操作のマッピングファイルの雛形を生成する。
    \item \texttt{generate-doc}: マニュアルを解析し,API 仕様と操作フローを含む中間表現 (IR) を生成する。
    \item \texttt{generate-api}: OpenAPI Generator を呼び出し,Flask サーバのスタブコードを生成する。
    \item \texttt{implement}: 生成されたスタブコードに対し,具体的な制御ロジックを注入する。
\end{itemize}

全体の実装プロセスは,開発者が行う準備フェーズと,ツールによる 3 つの自動生成フェーズ(解析・合成・注入)によって構成される。

\begin{enumerate}
  \item \textbf{Preparation (準備フェーズ)}:
  \texttt{init-bot-map} コマンドが担当する。
  開発者はまず本コマンドを実行して \texttt{switchbot.map.json} の雛形を生成し,
  物理的な SwitchBot デバイス(Bot)がどの家電のどのボタンに取り付けられているか(例: 照明の ON ボタン,炊飯器のスタートボタン)を手動で記述する。
  このマッピング情報は,後続のフェーズで LLM が操作対象を特定するための正解データとして機能する。

  \item \textbf{Analyzer (解析フェーズ)}:
  \texttt{generate-doc} コマンドが担当する。
  入力されたマニュアルを LLM (OpenAI API) を用いて解析し,
  操作手順を抽出して中間表現 (IR) を生成する。
  ここでは,幻覚 (Hallucination) のリスクを排除するため,
  プログラムコードそのものではなく,厳密なスキーマを持つ JSON データのみを出力させる。
  
  \item \textbf{Synthesizer (合成フェーズ)}:
  \texttt{generate-api} コマンドが担当する。
  OpenAPI Generator を用いてサーバサイドのスタブコード (Flask) を生成する。
  この際,独自にカスタマイズした Mustache テンプレートを適用することで,
  後続の処理でロジックを注入するための「マーカー (Marker)」をコード内に埋め込む。
  
  \item \textbf{Injector (注入フェーズ)}:
  \texttt{implement} コマンドが担当する。
  生成されたスタブコードに対し,IR に含まれる操作フロー定義に基づいて
  具体的な制御ロジックを決定論的に注入する。
  本研究では,この手法を \textbf{「マーカーベース注入 (Marker-based Injection)」} と呼び,
  構文エラーのない安全な実装を実現する核となる技術として位置づけている。
\end{enumerate}

次節以降では,これら各フェーズにおける詳細な実装ロジックについて述べる。
