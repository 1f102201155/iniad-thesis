\chapter{実装}
\label{chap:implementation}

本章では,第4章で設計した自動生成システム「Auto-IoT」の具体的な実装詳細について述べる。
本システムは TypeScript (Node.js) で開発された CLI ツールであり,
LLM の推論制御からコード生成,ロジック注入までを一貫して実行するよう実装されている。

以下では,まず開発環境とプロジェクト構成について述べ,
次にシステムの中核となる実装アプローチの概要を説明する。
その後,マニュアル解析,コード生成,ドライバ実装の各フェーズにおける詳細な実装ロジックを,
実際のソースコードを示しながら解説し,最後に実際の実行プロセスと生成物の変遷について述べる。

\section{開発環境と技術選定}
\label{sec:impl_env}

本システムは,開発効率と型安全性,および非同期処理の扱いやすさを考慮し,
メインの CLI ツールを TypeScript (Node.js) で実装した。
また,生成されるサーバサイドコードのターゲットとして,IoT 分野で広く利用され,
ライブラリのエコシステムが豊富な Python (Flask) を採用した。

具体的な開発環境および使用した主要ライブラリのバージョンを Table~\ref{tab:dev-env} に示す。

\begin{table}[h]
  \centering
  \caption{開発環境と主要ライブラリ}
  \label{tab:dev-env}
  \begin{tabular}{|l|l|p{6.5cm}|}
    \hline
    カテゴリ & 名称 & バージョン・備考 \\ \hline
    \textbf{Runtime} & Node.js & v23.5.0 \\ \hline
    \textbf{Language} & TypeScript & v5.8.3 \\ \hline
    \textbf{Core Libs} & commander & v13.1.0 (CLI フレームワーク) \\ \cline{2-3}
     & openai & v4.80.0 (LLM API クライアント) \\ \cline{2-3}
     & pdf-parse & v1.1.1 (PDF テキスト抽出) \\ \cline{2-3}
     & @openapitools/\newline openapi-generator-cli & v2.16.3 (Java 版 Generator のラッパー) \\ \hline
    \textbf{Target} & Python & 3.12 (生成されたコードの実行環境) \\ \cline{2-3}
     & Flask & 3.1.0 (Web フレームワーク) \\ \hline
  \end{tabular}
\end{table}

特に \texttt{@openapitools/openapi-generator-cli} は,
内部で Java 製の OpenAPI Generator (v7.17.0) をダウンロードして実行する仕組みとなっており,
これを利用することで最新の OpenAPI 仕様 (v3.0.3) に準拠したコード生成を実現している。
  % 開発環境と技術選定
% \section{プロジェクト構成}
\label{sec:impl_structure}

本システムのソースコードおよびリソースのディレクトリ構成概要を Fig.~\ref{fig:dir-structure} に示す。
設計の関心事を分離するため,
ユーザーインターフェースとなる CLI 定義 (\texttt{src/cli}),
コアロジックとなるサービス層 (\texttt{src/app/services}),
およびコード生成用のテンプレート (\texttt{templates}) を明確に分割した構成としている。

\begin{figure}[tb]
\begin{verbatim}
auto-iot/
├── src/
│   ├── app/
│   │   └── services/       # ビジネスロジック層 (Core Logic)
│   │       ├── doc.ts        # Phase 1: マニュアル解析・IR生成 (Analyzer)
│   │       ├── generator.ts  # Phase 3: スタブ生成制御 (Synthesizer)
│   │       ├── implement.ts  # Phase 3: ロジック注入 (Injector)
│   │       └── init-bot-map.ts
│   ├── cli/                # プレゼンテーション層 (CLI Definitions)
│   │   ├── index.ts          # エントリーポイント
│   │   ├── cmd-generate-doc.ts
│   │   ├── cmd-implement.ts
│   │   └── ...
│   ├── domain/             # ドメイン定義
│   │   └── generator-registry.ts
│   ├── integrations/       # 外部システム連携
│   │   └── switchbot/        # SwitchBot 関連の型定義等
│   └── lib/                # 共通ユーティリティ
│       ├── openai.ts         # OpenAI API 呼び出しラッパー
│       ├── logger.ts         # ロギング (pino)
│       └── errors.ts         # エラー定義
├── templates/
│   └── python-flask/       # コード生成用 Mustache テンプレート
│       ├── controller.mustache
│       ├── __extras__/       # 生成後にコピーされる追加ファイル群
│       │   └── openapi_server/
│       │       └── integrations/
│       │           └── switchbot_driver.py # 共通ドライバ
│       └── ...
├── package.json
├── tsconfig.json
└── openapitools.json       # Generator 設定
\end{verbatim}
\caption{Auto-IoT システムのディレクトリ構成}
\label{fig:dir-structure}
\end{figure}

\subsection{モジュールの役割}
\begin{itemize}
  \item \textbf{src/app/services/}:
  システムの核となるロジックが集約されているディレクトリである。
  特に \texttt{doc.ts} は LLM との対話および中間表現の操作を行い,
  \texttt{implement.ts} は生成されたコードへのパッチ適用(ロジック注入)を担当する。

  \item \textbf{templates/python-flask/}:
  OpenAPI Generator が使用する Mustache 形式のテンプレートファイル群である。
  標準の Flask テンプレートをベースにしつつ,
  ロジック注入に必要なマーカーコメント(\texttt{\# [AUTOIOT-FUNCTION-BEGIN]} 等)が
  自動的に出力されるようカスタマイズを加えている。
  また,\texttt{\_\_extras\_\_} ディレクトリには,生成コマンド完了後に
  出力先へそのままコピーされる静的ファイル(SwitchBot ドライバ等)を配置している。
\end{itemize}
    % プロジェクト構成
\input{chapters/05_implementation/05_03_overview}     % 実装アプローチの概要(抽象)
\input{chapters/05_implementation/05_04_analysis}     % マニュアル解析とIR生成の詳細 (doc.ts)
\input{chapters/05_implementation/05_05_generation}   % コード生成とロジック注入の詳細 (generator/implement.ts)
\input{chapters/05_implementation/05_06_driver}       % サーバサイドドライバの実装 (Python)
\input{chapters/05_implementation/05_07_process}      % 実行プロセスと生成物の変遷(具体)
\input{chapters/05_implementation/05_08_summary}      % まとめ
