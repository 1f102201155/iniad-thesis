\section{まとめ}
\label{sec:impl_summary}

本章では,提案手法に基づく IoT API 自動生成システムの実装について述べた。
開発環境としては Node.js および Python を採用し,
マニュアル解析を行う Analyzer,仕様書とスタブを生成する Synthesizer,
および制御ロジックを注入する Injector の 3 つのモジュールを連携させることで,
非決定的な LLM の出力と決定論的なコード生成を統合したパイプラインを構築した。

また,具体的な家電操作シナリオを用いた動作検証により,
マニュアルの入力から物理デバイスの操作に至る一連のプロセスが,
人手による修正なしに実行可能であることを確認した。

次章では,本システムを用いて複数の家電マニュアルを対象とした生成実験を行い,
生成成功率や操作フローの抽出精度について定量的な評価を行う。
