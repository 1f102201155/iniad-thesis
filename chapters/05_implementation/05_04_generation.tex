\section{コード生成とロジック注入の実装}
\label{sec:impl_generation}

Phase 2 および Phase 3 にあたるコード生成とロジック注入は,
OpenAPI Generator によるスタブ生成機能と,本システム独自の注入ロジックを組み合わせることで実現されている。

\subsection{マーカー付きテンプレートによるスタブ生成 (Synthesizer)}
Synthesizer フェーズでは,OpenAPI Generator を呼び出し,
Python (Flask) のサーバコードを生成する。

通常の OpenAPI Generator は API 定義に基づいた空の関数(スタブ)のみを生成するが,
本システムでは独自に拡張した Mustache テンプレートを使用している。
このテンプレートには,以下のようなコメント形式のマーカー定義が含まれている。

\begin{verbatim}
# [AUTOIOT-FUNCTION-BEGIN start_washing]
# [AUTOIOT-FUNCTION-END start_washing]
\end{verbatim}

このマーカーは,生成される API コントローラ内の各ハンドラ関数に埋め込まれ,
後続の Injector が「どこにコードを挿入すべきか」を特定するためのアンカーとして機能する。

\subsection{決定論的なロジック注入 (Injector)}
Injector フェーズは注入モジュールに実装されており,以下の手順で動作する。

\begin{enumerate}
  \item \textbf{操作フローの解決}:
  LLM が生成した操作フロー定義を読み込み,
  各ステップのターゲットラベルからデバイスマップ上の物理デバイス ID を取得する。
  この際,デバイスの動作モード(\texttt{pressMode} 等)と操作内容の整合性も再確認される。

  \item \textbf{Python コードの合成}:
  解決された操作手順に基づき,SwitchBot ドライバを呼び出す Python コード文字列を合成する。
  この生成ロジックは完全なルールベース(ハードコード)であり,LLM は関与しない。
  そのため,インデントのズレや未定義変数の参照といった構文エラーの混入が起こらないようになっている。

  \item \textbf{ソースコードの書き換え}:
  生成済みのコントローラファイルを読み込み,文字列検索で前述のマーカー領域を特定する。
  特定された領域の中身を,合成した Python コードで置換することで,
  API サーバの実装を完了させる。
\end{enumerate}
