\section{マニュアル解析と IR 生成の実装}
\label{sec:impl_analysis}

Phase 1 にあたるマニュアル解析プロセスは,解析を担うモジュール(Analyzer)に実装されている。
長大なマニュアルから正確な操作情報を抽出するために,
本システムでは「分割・要約・統合」のアプローチを採用している。

\subsection{テキストの読み込みとチャンク分割}
まず,入力されたマニュアル (PDF/Text) をプレーンテキストに変換し,正規表現を用いて整形する。
LLM のコンテキストウィンドウ制約に対応するため,テキストは一定サイズ(本実装では最大 12,000 文字)のチャンクに分割される。
この際,文脈の分断を防ぐため,チャンク間には 800 文字のオーバーラップを設けている。

\subsection{プロンプトエンジニアリングと制約}
分割された各チャンクは OpenAI API (\texttt{gpt-4o} 等) に送信される。
本システムでは,LLM の出力を制御するために,以下の 4 段階の構造を持つプロンプトを動的に構築している。

\begin{enumerate}
  \item \textbf{System Prompt}: 
  LLM の役割を定義し,「物理ボタンの操作手順のみを抽出すること」「状態確認などは無視すること」を指示する。
  また,出力フォーマットとして厳密な JSON スキーマを提示する。
  
  \item \textbf{User Prompt (Intro)}: 
  デバイスマップ (\texttt{switchbot.map.json}) の内容を提示し,
  使用可能なボタンラベル (\texttt{VALID\_LABELS}) を定義する。
  ここで,「マップに存在しないラベルの使用」や「勝手な翻訳・言い換え」を明確に禁止する制約を与える。
  
  \item \textbf{Summarization Prompts}: 
  各チャンクに対して要約を実行させ,そのチャンクに含まれる操作手順を抽出させる。
  
  \item \textbf{Final Instruction}: 
  全チャンクの要約結果を統合し,最終的な中間表現 (IR) を単一の JSON オブジェクトとして出力させる。
\end{enumerate}

\subsection{中間表現の構造とバリデーション}
LLM が生成した IR は,\texttt{info}, \texttt{operations}, \texttt{flow} の 3 つの要素を持つ JSON データである。
生成後,システムは直ちにバリデーションを実行する。
ここでは,IR 内の \texttt{targetLabel} がデバイスマップに実在するか,
またデバイスの動作モード (\texttt{pressMode} / \texttt{switchMode}) と操作タイプ (\texttt{press} / \texttt{on, off}) に矛盾がないかを検証する。
この検証プロセスにより,物理的に実行不可能な操作が含まれるリスクを未然に防いでいる。
