\section{開発環境と技術選定}
\label{sec:impl_env}

本システムは,開発効率と型安全性,および非同期処理の扱いやすさを考慮し,
メインの CLI ツールを TypeScript (Node.js) で実装した。
また,生成されるサーバサイドコードのターゲットとして,IoT 分野で広く利用され,
ライブラリのエコシステムが豊富な Python (Flask) を採用した。

具体的な開発環境および使用した主要ライブラリのバージョンを Table~\ref{tab:dev-env} に示す。

\begin{table}[h]
  \centering
  \caption{開発環境と主要ライブラリ}
  \label{tab:dev-env}
  \begin{tabular}{|l|l|p{6.5cm}|}
    \hline
    カテゴリ & 名称 & バージョン・備考 \\ \hline
    \textbf{Runtime} & Node.js & v23.5.0 \\ \hline
    \textbf{Language} & TypeScript & v5.8.3 \\ \hline
    \textbf{Core Libs} & commander & v13.1.0 (CLI フレームワーク) \\ \cline{2-3}
     & openai & v4.80.0 (LLM API クライアント) \\ \cline{2-3}
     & pdf-parse & v1.1.1 (PDF テキスト抽出) \\ \cline{2-3}
     & @openapitools/\newline openapi-generator-cli & v2.16.3 (Java 版 Generator のラッパー) \\ \hline
    \textbf{Target} & Python & 3.12 (生成されたコードの実行環境) \\ \cline{2-3}
     & Flask & 3.1.0 (Web フレームワーク) \\ \hline
  \end{tabular}
\end{table}

特に \texttt{@openapitools/openapi-generator-cli} は,
内部で Java 製の OpenAPI Generator (v7.17.0) をダウンロードして実行する仕組みとなっており,
これを利用することで最新の OpenAPI 仕様 (v3.0.3) に準拠したコード生成を実現している。
