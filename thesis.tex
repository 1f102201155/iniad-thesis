\documentclass[bachelor]{INIAD}%卒論用 
\addtolength{\footskip}{8mm}
\bibliographystyle{junsrt} 
%\usepackage[dviout]{graphicx}
\usepackage[dvipdfmx]{graphicx}
\usepackage{bm}
\usepackage{amsmath}
\usepackage{ascmac}
\usepackage[dvipdfmx]{hyperref}
\usepackage{pxjahyper}
\usepackage[inline,shortlabels]{enumitem}

% 論文作成時にToDoなどのメモを残すためのもの。後で消すこと
% --- draftトグル:最終版では \draftfalse にする ---
\newif\ifdraft
\draftfalse
\ifdraft
  % todonotes(すでに読み込み済みでなければこのままでOK)
  \usepackage{tikz}
  \usepackage[colorinlistoftodos,prependcaption,textsize=footnotesize]{todonotes}
  \setlength{\marginparwidth}{22mm}
  \newcommand{\todoinline}[2]{\todo[inline,color=#1!20]{#2}}
\else
  % 非ドラフト時は無効化
  \newcommand{\todoinline}[2]{}
\fi

% --- 状態バッジ(日本語・インライン) ---
\newcommand{\needrewrite}[1][]{\todoinline{orange}{\textbf{要書換}\ifx#1\empty\else:#1\fi}}
\newcommand{\needappend}[1][]{\todoinline{yellow}{\textbf{要追記}\ifx#1\empty\else:#1\fi}}
\newcommand{\needremove}[1][]{\todoinline{red}{\textbf{要削除}\ifx#1\empty\else:#1\fi}}
\newcommand{\needreview}[1][]{\todoinline{blue}{\textbf{要見直し}\ifx#1\empty\else:#1\fi}}

% 参考文献がまだのとき
\newcommand{\citenote}[1][]{\todoinline{purple}{\textbf{要引用}\ifx#1\empty\else:#1\fi}}

% 図・表・数式のプレースホルダ
\newcommand{\figtodo}[1][]{\todoinline{gray}{\textbf{図の準備}\ifx#1\empty\else:#1\fi}}
\newcommand{\tabtodo}[1][]{\todoinline{gray}{\textbf{表の準備}\ifx#1\empty\else:#1\fi}}
\newcommand{\eqtodo}[1][]{\todoinline{gray}{\textbf{数式の整備}\ifx#1\empty\else:#1\fi}}

% 章・節の粒度で未完をマーク
\newcommand{\sectodo}[1][]{\todoinline{brown}{\textbf{この節は下書き}\ifx#1\empty\else:#1\fi}}
% --- ここまでTodo用 ---

%\usepackage{geometry}
%\geometry{left=30mm,right=30mm,top=35mm,bottom=30mm}

%\documentclass[oneside]{suribt}% 本文が * ページ以下のときに (掲示に注意)
\title{LLMとOpenAPI Generatorを用いた家電向けIoT APIの自動生成}
%\titlewidth{}% タイトル幅 (指定するときは単位つきで)
\author{大友 裕太}
\eauthor{Yuta Otomo}% Copyright 表示で使われる
\studentid{1F10220115}
\supervisor{矢代 武嗣 教授}% 1つの引数をとる (役職まで含めて書く)
%\supervisor{指導教員名 役職 \and 指導教員名 役職}% 複数教員の場合,\and でつなげる
\handin{2026}{1}% 提出月. 2 つ (年, 月) 引数をとる
%\keywords{キーワード1, キーワード2} % 概要の下に表示される
\renewcommand{\baselinestretch}{1.25}
\setcounter{tocdepth}{2}

\begin{document}
\mojiparline{40}
\maketitle%%%%%%%%%%%%%%%%%%% タイトル %%%%

\frontmatter% ここから前文

%\etitle{Title in English}

%\begin{eabstract}%%%%%%%%%%%%% 概要 %%%%%%%%
% 300 words abstract in English should be written here. 
%\end{eabstract}

\begin{abstract}%%%%%%%%%%%%% 概要 %%%%%%%%
\needrewrite
本研究では、IoTデバイスの操作APIを自動生成するためのシステムを提案・実装した。
近年、家庭用や産業用のIoTデバイスが普及し、各機器に対応するAPIを個別に設計・実装する必要が生じている。
しかし、従来の手法では、製品マニュアルをもとに開発者が手作業でAPI仕様書を作成し、コードを実装するため、多大な工数と人為的誤りのリスクが存在した。

本研究では、大規模言語モデル(Large Language Model; LLM)とOpenAPI Generatorを組み合わせることで、
家電製品のマニュアルからAPI仕様および実装コードを自動生成するパイプラインを構築した。
具体的には、SwitchBot製デバイスを対象に、マニュアル文書とデバイスマップ(\texttt{switchbot.map.json})を入力とし、
LLMにより中間表現(Intermediate Representation; IR)を生成する。
このIRからOpenAPI仕様書(YAML/JSON)およびデバイス制御フロー定義を自動的に合成し、さらにOpenAPI GeneratorによりFlaskサーバのスタブコードを生成する。最終的に、生成された制御フローに基づいてSwitchBot制御コードを自動挿入することで、完全なIoT APIサーバを生成可能とした。

評価実験では、生成精度・再現性・生成時間・実行可能性などの観点から本手法を検証した。
その結果、マニュアル内容に対応するAPI仕様を高精度に生成でき、開発者が行う作業を大幅に削減できることを確認した。
本システムは、将来的に他のIoTデバイスやメーカー製品への拡張も可能であり、マニュアルベースのAPI自動生成における有効なアプローチとなることを示した。
\end{abstract}

%%%%%%%%%%%%% 目次 %%%%%%%%
{\makeatletter
\let\ps@jpl@in\ps@empty
\makeatother
\pagestyle{empty}
\tableofcontents
\clearpage}

\mainmatter% ここから本文 %%% 本文 %%%%%%%%

\chapter{はじめに}

\section{研究背景}
近年,IoT (Internet of Things) 機器の普及に伴い,スマートホーム市場が急速に拡大している。
総務省の調査によれば,世界のIoTデバイス数は年々増加傾向にあり,
家庭内においても照明,エアコン,掃除機,スマートロックなど多種多様な機器がネットワークに接続されるようになった\cite{SoumuIoT2023}。

これに伴い,家電の操作インターフェースは従来の物理的なリモコンやスイッチから,
スマートフォンアプリやスマートスピーカーを介したソフトウェア制御へと移行しつつある。
ユーザはインターネットを経由して外出先から家電の状態を確認したり,
複数の機器を連携させた自動制御(ホームオートメーション)を行ったりすることで,
生活の利便性とエネルギー効率を飛躍的に向上させることが可能となった。

しかし,市場の拡大は同時にシステム環境の複雑化も招いている。
現在普及している多くの IoT 家電は,メーカーごとに独自のクラウドサービスや通信仕様(プロトコル)を採用しており,
\textbf{互いに独立し,連携が困難な状態にある}。
家庭内には異なるメーカーの機器が混在し,これらを統一的に管理・制御するためには,
各機器が提供する API (Application Programming Interface) を適切に利用し,
システム間で連携させる技術が不可欠となっている。

\subsection{本研究が対象とするIoTシステム構成}
図\ref{fig:iot-basic-structure}に,スマートホームを含む一般的なIoTシステムの基本構成を示す。
左側上段は \textbf{ネットワークへ直接接続可能なデバイス}(スマートフォンやPC,セルラー/Wi-Fi内蔵機器)であり,
これらはクラウドと直接通信してサービスを利用する。
左側下段は \textbf{ネットワークへ直接接続しない(できない)デバイス}で,BLEやZigbee,赤外線などローカル無線で
\textbf{ゲートウェイ}(ハブ)に接続し,ゲートウェイがセルラーやブロードバンド経由でクラウドへ中継する。
中央の\textbf{ネットワーク}(インターネット/モバイル網)を介して,
右側の\textbf{サーバ}(クラウド)では認証・認可,レート制限,ログ収集,状態管理などの運用機能が実装される。

本研究では,この一般構成を前提に,\emph{クラウドAPI層とデバイス/ハブ層の間}でのインタフェース記述(OpenAPI)と
APIサーバ実装の自動生成に着目する。

\begin{figure}[tb]
  \centering
  \includegraphics[width=.95\linewidth]{figures/iot-basic-structure.jpg}
  \caption{IoTの基本構成(直結デバイス/非直結デバイス+ゲートウェイ/ネットワーク/サーバ)。
  図は Seeds4biz「【徹底解説】IoTを構成する技術要素|身体の機能に置き換えて考えてみよう」 の解説ページより引用\cite{SeedsIoTComponents}。}
  \label{fig:iot-basic-structure}
\end{figure}

しかし,IoT システムの構築と運用には,依然として高い技術的障壁が存在する。
これらを統合して動作させるためには,物理層からアプリケーション層までを繋ぎ合わせるフルスタックな実装スキルが要求される。
特に,ハードウェアに依存した通信仕様やデータ形式(バイナリデータや独自プロトコル)の理解は,
一般的な Web アプリケーション開発者にとって学習コストが高く,
これが新規サービス開発やシステム統合の参入障壁となっている。

\section{現状の課題}
IoT 機器の相互運用性(Interoperability)を確保するために,
ECHONET Lite や Matter といった標準規格の策定が進められている\cite{ECHONETSpec,MatterSpec}。
これらの規格に準拠した機器であれば,共通のインターフェースを通じて操作が可能である。

しかし現実には,標準規格に対応していないレガシーな家電や,
コスト制約により独自の簡易な通信仕様を採用した製品が市場には数多く存在する\cite{SpaceCore2024}。
また,標準規格に対応している機器であっても,メーカー独自の機能(拡張仕様)については
標準プロトコルでカバーしきれない場合が多い。

こうした「規格外」の機器をシステムに統合するためには,
開発者が各製品の取扱説明書や技術マニュアル(仕様書)を読み込み,
操作に必要なコマンド体系やパラメータを一つ一つ理解し,
それらをプログラムコード(ドライバ)として実装する必要がある\cite{SmartDev2024}。
この「マニュアル読解から実装まで」のプロセスは,
対象機器が増えるたびに繰り返す必要があり,極めて労働集約的である。
また,記述の曖昧な自然言語のマニュアルから正確な仕様を読み取る作業は,
ドメイン知識を持たない開発者にとっては困難であり,実装ミスやバグの温床となりやすい。

\section{本研究のアプローチ}
本研究では,大規模言語モデル(Large Language Model; LLM)の高い自然言語処理能力と
コード生成能力を活用することで,IoT 機器統合における「専門知識の壁」を取り除くことを目指す。
具体的には,IoT 機器の取扱説明書(PDF)を入力とするだけで,
その機器を操作するための Web API(OpenAPI 仕様)と,
実際に動作するサーバ実装コードを全自動で生成するシステムを提案する。

提案システムは,マニュアルから操作手順や通信仕様を抽出し,
IoT 制御に特化した中間表現(Intermediate Representation; IR)を経て,
標準的な REST API 定義である OpenAPI Specification (OAS) を生成する。
さらに,OpenAPI Generator を用いてサーバサイドのコード(Flask)を生成することで,
開発者はマニュアルを用意するだけで,即座に機器を Web API 経由で操作可能となる。

\section{本研究の目的と成果}
本研究の主な目的と貢献は以下の 2 点である。

\begin{enumerate}
  \item \textbf{SwitchBot Bot を介した物理操作によるレガシー家電の IoT 化手法の確立}:
  API を持たない物理デバイスに対し,SwitchBot Bot による物理的なボタン操作を抽象化・統合するシステムを構築した。
  これにより,既存の家電を買い替えることなく,あらゆる機器を即座にスマートホームの一部として制御可能であることを実証した。

  \item \textbf{LLM を用いた操作仕様の構造化と自動生成パイプラインの構築}:
  取扱説明書(非構造データ)から操作仕様を抽出し,OpenAPI 形式および制御コードへ変換する一貫した自動化パイプラインを開発した。
  本手法により,組み込み等の専門知識を持たない開発者でも容易に API を生成可能とした。
  さらに,本手法が IoT 開発のプロトタイピングにおける工数削減と構文的信頼性の確保に有効であることを明らかにした。
\end{enumerate}

\section{論文の構成}
本論文の構成は以下の通りである。
第2章では,IoT の相互運用性に関する既存技術および LLM を用いた仕様生成に関する関連研究について述べる。
第3章では,提案手法のコンセプトである,マニュアルから API 仕様を自動生成するアプローチの概要について説明する。
第4章では,提案手法を実現するためのシステム設計について述べ,データフローや各コンポーネントの役割を定義する。
第5章では,設計に基づいた実装の詳細について,使用したライブラリや環境を含めて述べる。
第6章では,複数の実際の家電マニュアルを用いた評価実験を行い,生成成功率,操作フローの整合性,および生成の一貫性について検証する。
第7章では,実験結果に基づいた考察を行い,本手法の有効性と限界,および今後の課題について議論する。
最後に第8章で本研究を総括し,結論を述べる。
    % 第1章 序論
\chapter{関連研究}

本章では,本研究の背景となる既存のアプローチと関連技術について整理する。
まず,(1) IoT システムにおける相互運用性確保の従来手法とその課題について述べ,
その解決策としての (2) LLM による API 仕様生成技術,
および (3) OpenAPI 仕様を用いたコード生成技術について概説する。
さらに,本研究で操作対象として採用する (4) 物理ボタン操作デバイスの制御技術についても触れる。
最後に,これらを踏まえた本研究の立ち位置について述べる。

% ---------------------------------------------------------
\section{IoTにおける相互運用性とインターフェース統合}

本研究の目的は多様な IoT 機器のインターフェースを統一的に扱うこと、
ひいては IoT 開発を高度な知識なく、簡単に実装可能にすることにある。
LLM を用いた自動生成のアプローチを論じる前に、
IoT の相互運用性(Interoperability)確保に関する従来のアプローチとその限界について整理する。

\subsection{標準化プロトコルによるアプローチ}
IoT 機器の断片化(Fragmentation)を解消するため、
通信プロトコルやデータモデルを標準化する取り組みが長年行われてきた。
国内においては ECHONET Lite、国際的には Matter などの
標準規格が策定されている\cite{ECHONETSpec,MatterSpec}。
これらの規格に準拠した機器同士であれば、事前の統合作業なしに相互操作が可能となる。

しかし、これらの標準化アプローチには「普及の壁」と「独自機能の欠落」という課題がある。
市場には標準規格に対応していないレガシー機器や、
コストの観点から独自の簡易なプロトコルを採用した機器が依然として多数存在する。
また、標準規格は汎用的な機能(ON/OFF や温度設定など)には対応できるが、
各メーカーが差別化のために搭載した独自機能(例:特定の快眠モードや AI 制御など)は
標準仕様の範囲外となり、結局は独自 API を利用せざるを得ないケースが多い\cite{SpaceCore2024}。

\subsection{IoT プラットフォームと手動統合}
プロトコルが異なる機器を統合するために、
Home Assistant や Eclipse SmartHome といった IoT プラットフォームが広く利用されている。
これらは「Binding」や「Integration」と呼ばれるドライバモジュールを介すことで、
HTTP、MQTT、Bluetooth などの異なるプロトコルを抽象化し、統一的な操作を可能にする。

しかし、これらの統合モジュールの開発は、コミュニティの開発者による手動実装に強く依存している。
新しい機器が発売されるたびに、専門知識を持つ開発者がマニュアルや通信パケットを解析し、
個別にコードを記述する必要があるため、
新製品への追従性やメンテナンスコストが大きなボトルネックとなっている\cite{SmartDev2024}。
この「手動実装」の壁が、IoT 開発の民主化を阻む要因となっている。

\subsection{Web of Things (WoT) と記述モデル}
W3C が推進する Web of Things (WoT) は、
Thing Description (TD) と呼ばれる JSON-LD 形式のメタデータを用いて
機器のインターフェースを記述し、Web 技術標準での統合を目指している。
WoT のアプローチは、機器の実装(プロトコル)とインターフェース記述を分離する点で
本研究の OpenAPI によるアプローチと親和性が高い。

しかし、既存の機器に対して適切な Thing Description を付与するためには、
やはり人間が仕様を理解し、記述を作成する必要がある。
本研究は、この「仕様記述の作成」という最も専門性を要する工程を、
LLM によるマニュアル読解によって自動化する試みとして位置づけられる。

% ---------------------------------------------------------
\section{LLMによる API 仕様生成}

前節で述べた「仕様記述コスト」の問題に対し、
近年では LLM を用いて API 仕様(OpenAPI)を自動生成する研究が活発に行われている。
本節ではオンラインドキュメント、コードなど異なる入力から OpenAPI を生成する手法を概説する。

\subsection{OASBuilder: Online API Documentation からの仕様生成}
OASBuilder\cite{OASBuilder2025} は,オンラインで公開されている API ドキュメントを入力として,
対応する OpenAPI 仕様書を自動生成することを目的とした手法である。
同一著者らによる先行研究 SpeCrawler\cite{SpeCrawler2023} も API ドキュメントを対象に
OpenAPI 仕様を生成する試みであったが,抽出結果の構造化や検証工程が限定的であり,
大規模な仕様では抜け漏れが生じやすいという課題があった。
OASBuilder はこれらの課題を踏まえ,より体系化された中間表現(IR)および検証フェーズを導入することで,
精度と信頼性を向上させている。

また、多くの商用 API ドキュメントは人間の閲覧を前提としており,HTML 構造や記述形式が統一されていないため,
LLM を直接適用すると項目の抜け漏れや不整合が生じやすい。
これに対して OASBuilder は,まず API ドキュメントから API 名称,パラメータ,レスポンス要素などを段階的に抽出し,
整理された中間表現(Intermediate Representation; IR)として統合する。
その後,IR を基に OpenAPI 仕様へ変換し,さらに生成結果を検証・修正する複数段階のパイプラインを採用している。

このように,長大で階層性の高い仕様情報を一度に生成せず,
\textbf{段階的抽出(decomposition)→構造化(IR)→仕様生成→検証}という逐次的な設計を導入することで,
LLM が不完全な情報から誤った JSON 構造を生成してしまう問題を緩和している。
特に IR の導入は,仕様生成を安定化させる上で重要な役割を果たし,
後続の整合性チェックやエラー修正も容易にする。
この点は,本研究において家電マニュアルとデバイスマップから IR を形成し,
OpenAPI Generator によるコード生成につなげるという構成とも共通しており,
OASBuilder は本研究のアプローチに近い先行研究といえる。

\subsection{LRASGen: ソースコードからの OpenAPI 生成}
LRASGen\cite{LRASGen2025} は,部分的なソースコードから RESTful API の振る舞いを推定し,
対応する OpenAPI 仕様を自動生成することを目的とした手法である。
既存の多くの API 仕様生成手法は,API ドキュメントやコメントを前提としており,
コード断片のみからエンドポイントやパラメータ,レスポンス構造を復元することは困難である。
LRASGenはこの課題に対し、ソースコードからエンドポイントメソッド,エンドポイントパラメータ,
パラメータ制約,エンドポイントレスポンスという四つのエンティティを段階的に同定し,
その結果に基づいて OpenAPI 仕様を生成するパイプラインを提案している。

LRASGen はコードベースのサービスに対して有効であり,
既存コードから API 仕様を後付けで構築する。
一方で,本研究が対象とする IoT 家電では公開ソースコードが提供されない場合が多く,
仕様生成はマニュアル記述や動作仕様など非構造的な情報から行う必要がある。
この点で LRASGen は入力ソースの性質が大きく異なり,
本研究のように説明文・機能一覧・デバイスマップから中間表現(IR)を構築し,
OpenAPI 仕様へ変換するアプローチとは方向性が異なる。
しかし,LRASGen が採用する「段階的抽出と LLM による仕様補完」という構成は,
構造化されていない入力から仕様を推定する際の一般的な設計指針として参考になる。

% ---------------------------------------------------------
\section{OpenAPI仕様からのコード生成}
巨大な OpenAPI 仕様を扱う研究として、LLM によるコード生成の自動化が検討されている。  
本研究では openapi-generator を用いてコード生成を行うため、LLM でコードまで生成する手法とは異なる立場を取る。


\subsection{LLMによる大規模 OpenAPI のコード生成}
大規模な OpenAPI 仕様からコード生成を行う研究として,
Pejcz らによる研究\cite{LargeOpenAPI_CodeGen2023}がある。
この研究では,コンテキスト長を超える規模の OpenAPI 仕様を
そのまま LLM に入力するのではなく,仕様を分割しつつ
プロンプトをオーケストレーションすることで,
REST API 実装コードを段階的に生成する手法を提案している。
本研究とは異なり,仕様生成からコード生成までを LLM が一貫して担う点に特徴があり,
OpenAPI Generator を用いて決定的にコード生成を行う本研究とは
役割分担の設計が異なる。

% ---------------------------------------------------------
\section{物理ボタン操作デバイスと制御 API}
\label{sec:switchbot_background}

本研究では,レガシー家電を含む多様な機器を物理的に制御する手段として,
SwitchBot 社のスマートホームデバイスを採用する。
本節では,そのハードウェア特性と制御 API の仕様について述べる。

\subsection{SwitchBot Bot の概要}
SwitchBot Bot(以下,Bot)\ref{fig:switchbot_bot_image}は,家電の物理スイッチやボタン付近に貼り付けて使用する
小型のロボットアームデバイスである\cite{SwitchBotBot}。
Bluetooth Low Energy (BLE) 通信によりスマホアプリやハブから制御され,
アームを動かして物理的にボタンを押下することで,
通信機能を持たない既存の家電(照明,炊飯器,給湯器など)を IoT 化することができる。

Bot には以下の 3 つの動作モードが存在する。
\begin{itemize}
  \item \textbf{プレスモード (Press Mode)}:
  「押して戻る」動作を行うモード。
  状態を持たないプッシュボタン(例: 炊飯器のスタートボタン)の操作に適している。
  \item \textbf{スイッチモード (Switch Mode)}:
  専用パーツを用いて「押す(ON)」と「引き上げる(OFF)」を行うモード。
  物理的な状態を持つロッカースイッチ(例: 壁の照明スイッチ)の操作に適している。
  \item \textbf{カスタムモード (Customize Mode)}:
    ON および OFF の操作コマンドに対して,それぞれ任意の動作シーケンス(アクションセット)を定義可能なモードである。
    1 つのアクションは複数の「ステップ」から構成され,各ステップには「押下時間」,「待機間隔」,「繰り返し回数」を指定することができる。
    これにより,長押しや連打といった複雑な物理操作パターンを実現する。
\end{itemize}

\begin{figure}
    \begin{center}
        \includegraphics[scale=0.2]{figures/SwitchBot_Bot_image.jpg}
    \end{center}
    \caption{SwitchBot Botの画像(SwitchBot Bot販売ページより引用)}
    \label{fig:switchbot_bot_image}
\end{figure}


\subsection{SwitchBot Cloud API}
SwitchBot 社は,同社製品をインターネット経由で制御するための Web API (SwitchBot Cloud API) を公開している\cite{SwitchBotCloudAPI}。
開発者は HTTP リクエスト(GET/POST)を送信することで,
デバイスの状態取得やコマンド送信を行うことができる。

Bot に対する制御コマンドは「操作」に特化しており,
「指定した ID の Bot を押す」という単純な命令を実行する。
ただし,Bot は物理的にボタンを押すことはできるが,
その結果として「家電が実際に動作したか(例: 炊飯が始まったか)」を検知するセンサは持たないため,
API からの制御は基本的に一方的なコマンド送信となる特性がある。

% ---------------------------------------------------------
\section{本研究との位置づけ}

以上の既存研究を踏まえると、以下の特徴が本研究の独自性につながる。

\begin{itemize}
  \item 家電マニュアルやデバイスマップから IR を生成し、OpenAPI 仕様を自動生成する点
  \item LLM は仕様生成(IR 形成)に限定し、コード生成は openapi-generator を利用するハイブリッド構成である点
  \item IoT デバイス操作に特化した API を自動生成し、実際の家電操作ドライバに統合する点
\end{itemize}

これらの点から、本研究は既存の仕様生成研究や IoT 自動化研究とは異なる独自の貢献を持つ。

   % 第2章 関連研究および技術的背景
\chapter{提案手法}
\label{chap:proposed_method}

本章では,家電マニュアルから IoT 操作用 Web API および制御コードを自動生成するための提案手法について述べる。
本研究の核となるのは,大規模言語モデル(LLM)の柔軟な読解能力と,従来のアルゴリズムによる堅牢なコード生成を組み合わせた「ハイブリッド生成アプローチ」である。

\section{システム概要}
提案システムの全体処理フローを図\ref{fig:system_flow}に示す。
本システムは,入力として「家電の取扱説明書(PDF/テキスト)」および「デバイスマップ(物理デバイスとラベルの対応表)」を受け取り,
以下の 3 段階の処理を経て,最終的に動作可能な API サーバのソースコードを出力する。

\begin{enumerate}
  \item \textbf{マニュアル解析と中間表現(IR)の生成}:
  LLM を用いてマニュアルから操作に関する情報のみを抽出し,本システム独自の軽量な JSON フォーマット(中間表現)に構造化する。
  \item \textbf{仕様書と制御フローの合成}:
  生成された IR を基に,決定論的なアルゴリズムを用いて標準的な OpenAPI 仕様書(OAS)と,デバイス操作手順書(Flow 定義)を自動合成する。
  \item \textbf{コード生成と実装注入}:
  OpenAPI Generator を用いてサーバのスタブコード(Flask)を生成し,その内部に Flow 定義に基づいた制御ロジックを注入する。
\end{enumerate}

この構成により,LLM 特有の「幻覚(Hallucination)」や「構文エラー」のリスクを最小限に抑えつつ,
多様なフォーマットで記述されたマニュアルへの対応を実現している。

\begin{figure}[tb]
  \begin{center}
    \includegraphics[width=0.5\textwidth]{figures/application_overview.png}
  \end{center}
  \caption{}
  \label{fig:system_flow}
\end{figure}

\section{中間表現 (Intermediate Representation; IR) の導入}
\label{sec:ir_design}

既存の多くの手法では,LLM に直接 OpenAPI 仕様(YAML/JSON)を出力させようとする。
しかし,OpenAPI 仕様は非常に冗長であり,かつ厳密なスキーマ構造(ネストされたオブジェクトや参照定義)を持つため,
LLM が一貫して正しい構文を出力することは困難である。
特に,トークン長制約の中で巨大な仕様書を生成させると,JSON の閉じ括弧の欠落や,必須フィールドの省略といったエラーが多発する。

さらに,本研究における予備検討として,LLM に直接 OpenAPI 仕様(YAML)を生成させるアプローチを試みたところ,
マニュアルの自然言語記述に含まれる「:(コロン)」が適切にエスケープされず,
パースエラーが多発するという実用上の課題も確認された。

そこで本手法では,LLM の出力を API 構築に必要な\textbf{「最小限の情報」}に限定するアプローチをとる。
これを\textbf{中間表現(IR)}と定義する。
IR は以下の 3 つの要素のみを持つシンプルな JSON オブジェクトである。

\begin{itemize}
  \item \textbf{info}: API のタイトル。
  \item \textbf{operations}: API として公開すべき操作のリスト(エンドポイントパス,HTTP メソッド,操作 ID)。
  \item \textbf{flow}: 各操作 ID に対応する具体的なデバイス制御手順(ボタン操作の順序)。
\end{itemize}

パラメータ定義やレスポンス定義,エラーハンドリングといった「定型的な」情報は IR には含めず,
後段のプログラムで自動補完する設計とすることで,LLM の負荷を下げ,生成の安定性を向上させている。

\section{ハイブリッド生成プロセス}

\subsection{Phase 1: LLM による抽出 (Extraction)}
このフェーズでは,自然言語で記述されたマニュアルから IR を生成する。
ここでの LLM の役割は,新たな操作手順を作り出す「創造」ではなく,
マニュアルから必要な情報を抜き出す\textbf{「抽出」}と,
自然言語を機械可読な形式へ変換する\textbf{「構造化(マッピング)」}に限定される。

具体的には,マニュアルをテキストチャンクに分割して LLM に入力し,
「どのボタンを,どういう順序で押せば,どのような操作が実現できるか」という情報を抽出させる。
この際,以下の制約をプロンプトで与えることで,出力の品質を担保する。

\begin{itemize}
  \item \textbf{操作の限定}: 電源 ON/OFF やモード変更など,物理ボタンで操作可能な機能のみを抽出対象とし,状態確認(Status Polling)はSwitchBot Botでは実現できないため対象外とする。
  \item \textbf{ラベルの厳密一致}: 操作対象となるボタン名は,後述するデバイスマップに定義されたラベル(Valid Labels)と完全に一致させることを強制する。未知のラベルや,LLM が想像で作ったラベルの使用は禁止する。
\end{itemize}

\subsection{Phase 2: 決定論的合成 (Deterministic Synthesis)}
Phase 1 で生成された IR は,TypeScript で実装されたジェネレータによって処理され,以下の 2 つの成果物に変換される。
この変換プロセスはルールベースのアルゴリズムであり,LLM は介入しないため,常に構文的に正しい出力が保証される。

\begin{enumerate}
  \item \textbf{OpenAPI Specification (openapi.json/yaml)}:
  IR の \texttt{operations} リストに基づき,パスやメソッドを定義する。
  この際,正常系(200 OK)および異常系(401 Unauthorized, 503 Service Unavailable 等)のレスポンススキーマは,
  システム側で用意したテンプレートを全エンドポイントに一律で適用する。
  これにより,手動で記述するには手間のかかる厳密な API 定義を瞬時に生成できる。

  \item \textbf{操作フロー定義 (switchbot\_flow.json)}:
  IR の \texttt{flow} 部分を切り出し,アプリケーションが実行時に参照するための設定ファイルとして保存する。
  ここでは,API の \texttt{operationId} と,実際の物理ボタン操作(\texttt{press}, \texttt{on}, \texttt{off})のマッピングが定義される。
\end{enumerate}

\subsection{Phase 3: コード生成 (Code Generation)}
生成された OpenAPI 仕様を入力として,OSS ツールである OpenAPI Generator を実行し,
Python (Flask) のサーバコードを生成する。
本研究では,生成されたコントローラコード(スタブ)に対し,
Phase 2 で生成したフロー定義を読み込んで実行する汎用的なドライバロジックを注入する。
これにより,開発者は一切のコーディングを行うことなく,マニュアルの内容を反映した IoT API サーバを手に入れることができる。

\section{デバイスマップによる物理抽象化}
第\ref{sec:switchbot_background}節で述べた通り,SwitchBot Bot と Cloud API を利用することで,
通信機能を持たないレガシー家電であっても,物理ボタンへの後付け設置によって API 経由での制御が可能となる。
本研究では,この「物理操作の API 化」という利点を活かすことで,
マニュアルから単に仕様書を作成するだけでなく,実際に物理デバイスを操作可能なシステムを構築する。

家電の種類やメーカーによって,操作に必要な SwitchBot デバイスの構成は異なる(例:物理ボタンを押す「ボット」,赤外線を飛ばす「ハブ」など)。
本手法では,これらの物理デバイス構成を \texttt{switchbot.map.json} という定義ファイルに抽象化して入力する。

しかし,マニュアルに記述されている「ボタン名(ラベル)」と,
実際の API 制御に必要な「デバイス ID」は直接対応していない。
そこで本手法では,両者を仲介する抽象化レイヤとして \textbf{「デバイスマップ」} を導入する。

\subsection{デバイスマップの定義構造}
デバイスマップ (\texttt{switchbot.map.json}) は,
家電の「機能的なラベル」と「物理的なデバイス情報」の対応関係を定義した JSON ファイルである。
このマップにより,以下の 2 点の抽象化を実現している。

\begin{enumerate}
  \item \textbf{ID の隠蔽とラベル解決}:
  LLM は「電源ボタン」「スタート」といった意味的なラベルのみを扱って操作フローを生成する。
  システムは実行時にこのマップを参照し,ラベルに対応する物理デバイス ID(例: \texttt{E16A82C65393})へと変換する。
  これにより,LLM の出力は特定のハードウェア環境に依存しない汎用的なものとなる。

  \item \textbf{動作モードの制約}:
  各デバイスに対して,第\ref{sec:switchbot_background}節で説明した Bot の動作モード
  (\texttt{pressMode} / \texttt{switchMode}) をマップ上で定義する。
  システムはこの定義に基づき,LLM が生成した操作タイプ(\texttt{press} なのか \texttt{on/off} なのか)が
  物理的に実行可能であるかを検証(バリデーション)する。
\end{enumerate}
 % 第3章 提案手法
\chapter{システム設計}
\label{chap:system_design}

本章では,前章で提案した手法を具現化する自動生成システム「Auto-IoT」の詳細設計について述べる。
本システムは TypeScript (Node.js) で実装された CLI ツールであり,
LLM と決定論的アルゴリズムを密に連携させることで,マニュアル入力からコード生成までを実行する。

\section{ソフトウェア・アーキテクチャ}
本システムは,図\ref{fig:module_architecture}に示すように,データの変換と生成を担う以下の 3 つのモジュールによって構成される。

\begin{enumerate}
  \item \textbf{Analyzer (解析モジュール)}:
  マニュアル PDF を読み込み,LLM を用いて中間表現 (IR) を生成する。
  ここではトークン制限への対策や,ハルシネーション抑制のための検証ロジックが含まれる。
  \item \textbf{Synthesizer (合成モジュール)}:
  IR を基に,OpenAPI 仕様書 (OAS) と操作フロー定義 (Flow) を決定論的に合成する。
  \item \textbf{Injector (実装注入モジュール)}:
  OpenAPI Generator によって出力されたサーバコードに対し,物理デバイス操作のためのドライバロジックを自動的に注入する。
\end{enumerate}

\begin{figure}[tb]
  \centering
  \resizebox{\linewidth}{!}{
    \begin{tikzpicture}[
      node distance=1.2cm and 1.8cm,
      module/.style={
        rectangle, 
        draw=black, 
        thick, 
        fill=gray!10, 
        text width=2.5cm, 
        minimum height=1.5cm, 
        align=center, 
        rounded corners=3pt,
        font=\bfseries
      },
      data/.style={
        rectangle, 
        draw=none, 
        text width=2.2cm, 
        align=center, 
        font=\small
      },
      arrow/.style={
        -{Stealth[length=3mm]}, 
        thick
      },
      lbl/.style={
        midway,
        above,
        font=\small,
        align=center
      }
    ]

      % --- Nodes ---
      % Inputs
      \node[data] (input) {Manual PDF\\Device Map};

      % Modules
      \node[module, right=of input] (analyzer) {Analyzer\\(解析)};
      \node[module, right=of analyzer] (synthesizer) {Synthesizer\\(合成)};
      \node[module, right=of synthesizer] (injector) {Injector\\(注入)};

      % Output
      \node[data, right=of injector] (output) {Server Code\\(Python)};

      % --- Arrows & Labels ---
      \draw[arrow] (input) -- node[lbl] {Input} (analyzer);
      \draw[arrow] (analyzer) -- node[lbl] {IR} (synthesizer);
      \draw[arrow] (synthesizer) -- node[lbl] {OAS / Flow} (injector);
      \draw[arrow] (injector) -- node[lbl] {Generate} (output);

    \end{tikzpicture}
  }
  \caption{提案システムのモジュール構成とデータフロー}
  \label{fig:module_architecture}
\end{figure}


以下,各プロセスの詳細設計について述べる。

\section{マニュアル解析プロセス (Phase 1)}
マニュアルは数百ページに及ぶ場合があり,
LLM のコンテキストウィンドウ(入力トークン制限)を超える可能性がある。
また,一度に大量の情報を入力すると,LLM の注目(Attention)が散漫になり,
重要な操作手順を見落とすリスクがある。
これに対処するため,本システムでは以下の「分割・要約・統合」アプローチを採用した。

\subsection{チャンク分割と要約}
入力されたマニュアルテキストを,
LLM のコンテキストウィンドウ制限に収まる一定サイズのチャンクに分割する。
単に分割しただけでは,境界部分で文脈が分断され,
重要な操作手順の情報が失われるリスクがある。
そのため,本手法ではチャンク間に一定のオーバーラップ(重複領域)を設けることで,
文脈の連続性を担保している。

各チャンクは LLM に順次入力され,
プロンプトによって「操作手順に関連する部分」のみが要約(Summarization)される。
本システムにおける要約の方針は以下の通りである。

\begin{itemize}
  \item \textbf{抽出対象}: 物理ボタンの操作手順(どのボタンを,どの順序で押すか)。
  \item \textbf{除外対象}: 状態確認(Status Polling),センサ連動などの SwitchBot Bot では実現不可なロジック。
\end{itemize}

\subsection{中間表現の生成と制約}
全チャンクの要約が完了した後,それらを統合して最終的な中間表現(IR)を生成する。
この際,システムプロンプトにおいて以下の制約を課すことで,出力の品質を担保している。

\begin{itemize}
  \item \textbf{役割定義}: 「あなたは家電マニュアルから操作手順のみを抽出するエンジニアである」と定義。
  \item \textbf{翻訳・言い換えの禁止}: ボタンラベルはデバイスマップ(\texttt{switchbot.map.json})に定義されたものと完全に一致させ,LLM による勝手な翻訳(例: "Power Button" $\to$ "電源ボタン")を禁止する。
  \item \textbf{出力形式の固定}: JSON 以外の解説文(Markdown のコードフェンス等)を含まない単一の JSON オブジェクトとして出力させる。
\end{itemize}

\section{仕様合成と検証 (Phase 2)}
LLM が生成した IR は,TypeScript で実装されたバリデータによって検証された後,OpenAPI 仕様へと変換される。

\subsection{バリデーションロジック}
生成された IR に対し,以下の整合性チェックを行う。

\begin{itemize}
  \item \textbf{ラベル存在確認}: IR 内で使用されているボタンラベルが,デバイスマップに実在するか。
  \item \textbf{モード整合性確認}: デバイスマップで「押す操作のみ (pressMode)」と定義されているボタンに対し,「ON/OFF 操作」を割り当てていないか。
\end{itemize}

これらのチェックにより,LLM が存在しないボタンを操作しようとしたり,
物理的に不可能な操作を定義したりするエラーを未然に防ぐ。

\subsection{テンプレート適用による仕様化}
検証を通過した IR は,テンプレートを用いて OpenAPI 仕様書(JSON/YAML)に変換される。
API の各エンドポイント(操作)に対し,正常系(200 OK)だけでなく,
認証エラー(401),デバイス未検出(404),サービス利用不可(503)などの
標準的なエラーレスポンス定義を一律で適用する。
これにより,LLM に記述させることなく,プロダクションレベルの堅牢な API 定義を自動生成している。

\section{コード生成とロジック注入 (Phase 3)}
最終的なサーバコードの構築は,OpenAPI Generator によるスタブ生成と,本システム独自のロジック注入によって行われる。

\subsection{スタブコード生成}
まず,生成された OpenAPI 仕様を入力として \texttt{openapi-generator-cli} を実行し,
Python (Flask) のサーバコードを生成する。
一般的なコード生成ツールは,仕様書からボイラープレート(雛形)を作成することには長けているが,
生成後のコードに動的にロジックを注入するための拡張性は考慮されていないことが多い。
そのため,デフォルトのテンプレートをそのまま使用すると,
後段の処理で正確な挿入位置を特定することが困難となる。

そこで本手法では,生成時に専用の Mustache テンプレートを適用し,
コントローラ(\texttt{default\_controller.py})内の各関数に,
後処理用の「マーカー」を埋め込む設計とした。
このマーカーは Python のコメント構文であるため実行に影響を与えず,
Injector が注入範囲を特定するための境界線として機能する。
図~\ref{fig:example_flow_compact}に示すように,各 API 関数の冒頭と末尾へ
\texttt{\# [AUTOIOT-FUNCTION-BEGIN ...]} という形式のコメント行が出力される。

\begin{figure}[tb]
  \centering
  \begin{tcolorbox}[
    width=\linewidth,
    colback=black!2!white,
    colframe=gray!60,
    boxrule=0.4pt,
    arc=2pt,
    left=2mm, right=2mm, top=1mm, bottom=1mm
  ]
  \inputminted[fontsize=\footnotesize, breaklines, tabsize=2]{python}{example_codes/marker.py}
  \end{tcolorbox}
  \caption{実際にマーカーが埋め込まれる例}
  \label{fig:example_flow_compact}
\end{figure}


\subsection{ドライバロジックの注入 (Implementation Injection)}
スタブ生成後,本システムの \texttt{Injector} モジュールがコントローラファイルを解析し,
マーカーで囲まれた領域に対して,操作フロー定義(\texttt{switchbot\_flow.json})に基づいた Python コードを自動生成して上書き(注入)する。
注入されるコードには,以下の機能がハードコードされた状態で埋め込まれる。

\begin{enumerate}
  \item \textbf{SwitchBot ドライバの呼び出し}:
  定義されたデバイス ID に基づき,\texttt{bot.press()} や \texttt{bot.turn\_on()} を順次実行する処理。
  \item \textbf{実行時のモード検証}:
  Python コードレベルでも `if mode == 'pressMode'` といった条件分岐を挿入し,定義と矛盾する操作が行われないよう実行時にもガードをかける。
  \item \textbf{エラーハンドリングの集約}:
  SwitchBot API が返す独自のエラーコード(例: 161 デバイスオフライン)や HTTP ステータスコードへの変換ロジックはドライバ層に集約し,
  注入するコードからは共通のエラー生成メソッドを呼び出すだけの簡潔な構成とする。
\end{enumerate}

この「マーカーベースの注入方式」を採用することで,
OpenAPI Generator が生成するボイラープレート(ルーティングや型定義)の恩恵を受けつつ,
中身のロジックだけを安全かつ確実に差し替えることが可能となっている。
   % 第4章 システム設計
\chapter{実装}

\section{開発環境}
Node.js, TypeScript, Python(Flask)による実装環境を記す。

\section{主要コンポーネント}
各コマンドおよびドライバ実装の概要を述べる。

\section{動作例}
代表的なAPI生成例を示す。
  % 第5章 実装
\chapter{評価実験}
本章では,本研究で提案した LLM と OpenAPI Generator を用いた
IoT 向け API 自動生成手法について,評価実験の設定および結果を示す。
まず実験環境および評価方針を述べた後,
評価指標を定義し,最後に GPT-4o と GPT-4o-mini による比較結果と考察を述べる。

\section{実験環境}
本節では,評価に用いたハードウェア構成,ソフトウェア構成,
LLM モデルと利用期間,ネットワーク環境,
および評価対象データについて述べる。

%---------- ハードウェア構成 ---------------
\subsection{ハードウェア構成}
評価実験には,表 ~\ref{tab:eval-hw} に示す構成の PC を用いた。
LLM 呼び出し,OpenAPI 仕様生成,コード生成はすべてこの環境で実行した。

\begin{table}[tb]
  \centering
  \caption{評価に用いたハードウェア構成}
  \label{tab:eval-hw}
  \begin{tabular}{|l|l|}
    \hline
    項目   & 内容 \\ \hline
    CPU   & 12th Gen Intel(R) Core(TM) i7-12650H \\ \hline
    GPU   & NVIDIA GeForce MX550 \\ \hline
    メモリ & 32\,GB \\ \hline
    OS    & Windows 11 \\ \hline
    実行環境 & PowerShell 7.5.4 \\ \hline
  \end{tabular}
\end{table}

%---------- ソフトウェア構成 ---------------
\subsection{ソフトウェア構成}
主要な開発環境およびミドルウェアのバージョンを
表 ~\ref{tab:eval-sw} に示す。
本研究で開発した CLI ツール(auto-iot)は Node.js 上で動作し,
LLM の呼び出しや OpenAPI Generator の起動を内部で行う。

\begin{table}[tb]
  \centering
  \caption{評価に使用した主なソフトウェア環境}
  \label{tab:eval-sw}
  \begin{tabular}{|l|l|}
    \hline
    項目 & バージョン・内容 \\ \hline
    Python & 3.12.10 \\ \hline
    Node.js & 23.5.0 \\ \hline
    パッケージマネージャ & npm(npx 経由で CLI 実行) \\ \hline
    OpenAPI Generator & 7.17.0\footnotemark[1] \\ \hline
    Auto-IoT CLI(本研究ツール) & version 0.1.0 \\ \hline
  \end{tabular}
\end{table}

\footnotetext[1]{以下のコマンドにより Java 版 OpenAPI Generator のバージョンを確認した。
\texttt{npx @openapitools/openapi-generator-cli version} の実行時に
バイナリが自動的にダウンロードされ,7.17.0 が選択されたことを確認している。}

%---------- LLMモデルおよび利用期間 ---------------
\subsection{LLM モデルおよび利用期間}
本研究では OpenAI 社の Chat Completions API を用いて LLM 呼び出しを行った。
評価対象としたモデルは以下の 2 つである。

\begin{itemize}
  \item GPT-4o
  \item GPT-4o-mini
\end{itemize}

API 呼び出しは 2025 年 10 月 27 日から 11 月 5 日までの期間に実施し,
この期間に提供されていた最新版のモデルを使用した。

%---------- 評価対象データと実行条件 ---------------
\subsection{評価対象データと実行条件}
評価対象として,10 種類の家電製品の取扱説明書(PDF ファイル)を収集した。
各マニュアルに対し,以下の手順で評価データを準備した。

また,本研究では各マニュアルに対して \textbf{3 回ずつ} API 生成処理(generate-doc)を実行した。
そのため,評価に用いた生成結果は \textbf{合計 30 件} となる。

\begin{enumerate}
  \item 各家電のマニュアル PDF を 1 件ずつ用意する。
  \item CLI サブコマンド \texttt{init-bot-map} により
        \texttt{switchbot.map.json} の雛形を生成する。
  \item appliance, button, mode などの項目を手動で補完し,
        マニュアルに記載されたボタン・操作と整合するよう編集する。
\end{enumerate}
  % 実験環境
\section{評価方針}
\label{sec:policy}

本研究の目的は,高度なドメイン知識を持たない開発者であっても,
マニュアルを用意するだけで IoT 機器の操作 API を容易に構築可能にすることである。
前節で述べた通り,本評価では物理デバイスを用いた実機検証は行わず,
生成されたソフトウェア成果物の品質と,生成プロセスの実用性に焦点を当てる。

具体的には,以下の 3 つの観点から定量的な評価を行う。

\subsection{生成物の機能的妥当性}
生成された成果物が,システムとして破綻なく機能するかを評価する。
LLM は時に構文的に誤った JSON やコードを出力する場合がある(ハルシネーション)。
本システムにおいては,中間表現から OpenAPI 仕様書を経てサーバーコードに至るパイプラインが
エラーなく完走し,最終的に Flask サーバーが正常に起動・待機状態になれるかを
「生成成功」の最低条件として定義する。

\subsection{マニュアル記述との整合性}
生成された API が,入力としたマニュアルの内容を正しく反映しているかを評価する。
サーバーが起動しても,操作コマンドが不足していたり,
誤ったパラメータが設定されていては実用性がない。
そのため,マニュアルに記載されている操作フロー(手順)が,
過不足なく API のロジックとして抽出されているかを,
正解データ(Ground Truth)との比較により検証する。

\subsection{コスト効率とモデル特性}
実用的なシステム運用において,API 利用コスト(トークン量)や生成時間は重要な要素である。
一般に,高性能モデルは高コスト・低速であり,軽量モデルは低コスト・高速であるとされる。
本実験では,GPT-4o と GPT-4o-mini の 2 つのモデルで生成を行い,
消費トークン数や所要時間を計測することで,
IoT ドライバ生成タスクにおいてどちらのモデルがコスト対効果(Cost-Performance)に優れているかを明らかにする。
       % 評価方針
\section{評価項目}

本研究では,提案手法によって自動生成される API 仕様およびコードの品質を多面的に検証するため、
以下の 6 つの指標を用いて評価を行う。これらの指標は,
仕様生成の正確性、手順整合性、コードの実行可能性、再現性、
および効率性を総合的に把握することを目的としている。

\begin{enumerate}
    \item \textbf{生成成功率}  
          自動生成処理がエラーなく最後まで完了し,生成された OpenAPI 仕様が構文的に正しいかを評価する指標である。
          本指標は,仕様生成パイプラインの安定性や堅牢性を確認することを目的とする。

    \item \textbf{操作フロー整合性}  
          生成された API 仕様および flow.json が,マニュアルに記載された操作手順とどの程度一致しているかを測る指標である。
          操作の過不足やボタンの対応関係,ステップ構造などの正確性を評価する。

    \item \textbf{アプリ起動性}  
          自動生成された Flask サーバコードが追加修正なしで起動可能かを評価する指標である。
          これはコード生成テンプレートおよび LLM が生成した operationId 一貫性の検証に対応する。

    \item \textbf{生成時間}  
          API 仕様生成(generate-doc)からコード生成(generate-api)の完了までに要する処理時間を評価する。
          モデルごとの処理速度や実運用における応答性の観点を把握する目的がある。

    \item \textbf{生成一貫性}  
          同一の入力(マニュアル PDF)に対して複数回生成を行った際に,
          LLM が生成する flow.json の構造がどの程度安定して再現されるかを評価する指標である。
          LLM 出力のゆらぎを測るために導入した。

    \item \textbf{トークン効率}  
          LLM API 呼び出しにおけるプロンプトトークン,応答トークンの消費量を比較し,
          モデルごとの効率性や運用コストへの影響を評価する指標である。
\end{enumerate}

各指標の算出結果を後述の Table~\ref{tab:eval-summary} に示す。
      % 評価項目
\section{評価方法}
本節では、前節で定義した6つの評価項目について、
実際にどのような基準と手順で評価を行ったかを説明する。

\subsection{生成成功率の評価方法}
生成成功率は、generate-docからimplementまでの一連のコマンドが成功するかを評価したものである。
以下のいずれかの条件がすべてを満たした場合を成功とした。

\begin{itemize}
  \item generate-doc が例外なく終了すること
  \item gneerate-api が例外なく終了すること
  \item 生成されたopenapi.yamlが構文的に正しいこと(YAML パーサで検証)
\end{itemize}

\subsection{操作フロー整合性の評価方法}
操作フロー整合性は、生成されたAPI仕様及び、
flow.jsonがマニュアルに記載された操作とどの程度一致しているかを測定する指標である。

\begin{enumerate}
  \item \textbf{操作 API の正誤判定(TP/FP/FN)}  
        マニュアルから手動で作成した正解集合(ゴールドセット)と比較し,
        必要な API が生成されたか(TP),
        不要な API が混入したか(FP),
        必要な API が欠落しているか(FN)を判定した。

  \item \textbf{構造整合性(Step Order Accuracy)}  
        TP と判定された API について,
        操作ステップ数・順序,targetLabel,操作タイプ(press/on/off)が
        マニュアルと一致しているかを評価した。

  \item \textbf{Precision・Recall・F1 によるモデル性能比較}  
        \[
        Precision = \frac{TP}{TP + FP},\quad
        Recall = \frac{TP}{TP + FN},\quad
        F1 = 2\frac{Precision \cdot Recall}{Precision + Recall}
        \]
        を算出し,GPT-4o と GPT-4o-mini の性能を比較した。
\end{enumerate}

\subsection{アプリ起動性の評価方法}
アプリ起動性は、生成された Flask サーバコードが  
追加修正なしに起動可能かどうかを評価するため、

\begin{itemize}
  \item python -m openapi\_server がエラーなく起動するかを確認
\end{itemize}

という基準で判定した。  

\subsection{生成時間の評価方法}

生成時間(Generation Time)は、以下の処理を連続で実行した際の経過時間を測定した。

\begin{itemize}
  \item generate-doc による IR 生成およびテンプレ合成
  \item generate-api によるコード生成
  \item implement によるコードの実装
\end{itemize}

各マニュアルについて3回ずつ測定し、計30件の結果から  
モデルごとの平均時間を算出した。

\subsection{生成一貫性(Generation Consistency)の評価方法}

生成一貫性は,\textbf{同一のマニュアル PDF に対して複数回 API 生成処理を実行したとき,
LLM によって生成される \texttt{flow.json} の構造がどの程度安定して再現されるか} を評価する指標である。

本研究では、1つのマニュアルに対して3回生成を行い、
得られた3件の \texttt{flow.json} を \(F^{(1)}, F^{(2)}, F^{(3)}\) とする。
このうち \(F^{(1)}\) を基準とし、\((F^{(1)}, F^{(2)})\)、\((F^{(1)}, F^{(3)})\) の2組のペアについて
構造的一致度スコアを計算し、その平均を当該マニュアルの生成一貫性スコアとした。
同様の手順を10種類のマニュアルに対して行い、その平均値をモデルの生成一貫性として用いる。

ペア \((F^{(a)}, F^{(b)})\) のスコアは、flow 単位の一致度を平均することで求める。
\(F^{(a)}\) に含まれる flow の集合を \(\mathcal{F}_a\)、\(F^{(b)}\) に含まれる集合を \(\mathcal{F}_b\) とし、
要素数をそれぞれ \(|\mathcal{F}_a|\), \(|\mathcal{F}_b|\) とおく。
このとき、ペアのスコア \(S_{\mathrm{pair}}(F^{(a)}, F^{(b)})\) は

\[
S_{\mathrm{pair}}(F^{(a)}, F^{(b)}) =
\frac{1}{N}
\sum_{f \in \mathcal{F}_a} S_{\mathrm{flow}}(f, \mathrm{match}(f)),
\quad
N = \max(|\mathcal{F}_a|, |\mathcal{F}_b|)
\]

で定義する。
ここで \(\mathrm{match}(f)\) は、\(\mathcal{F}_b\) のうち \(f\) と同じ operationId を持つ flow が存在する場合にはその flow、
存在しない場合には「欠落 flow」とみなしてスコア 0 を与えることを表す。

各 flow ペア \((f_1, f_2)\) のスコア \(S_{\mathrm{flow}}(f_1, f_2)\) は、
operationId の一致と steps 構造の一致度を統合して次のように定義する。

まず、operationId の一致に対して重み 0.4 を与える。

\[
S_{\mathrm{op}} =
\begin{cases}
0.4, & \text{if } \mathrm{operationId}(f_1) = \mathrm{operationId}(f_2) \\
0,   & \text{otherwise}
\end{cases}
\]

operationId が一致しない場合は根本的に別の操作とみなし、
\(S_{\mathrm{flow}}(f_1, f_2) = 0\) とする。

次に、\(f_1\) と \(f_2\) に含まれるステップ列をそれぞれ
\(\mathrm{steps}_1 = (\sigma^{(1)}_1, \dots, \sigma^{(1)}_{L_1})\),
\(\mathrm{steps}_2 = (\sigma^{(2)}_1, \dots, \sigma^{(2)}_{L_2})\) とする。
ここで \(\sigma^{(j)}_i\) は i 番目のステップ (type と targetLabel を含む構造) を表す。
比較に用いる長さ \(L\) を

\[
L = \max(L_1, L_2)
\]

とし、先頭から \(\min(L_1, L_2)\) 個のステップを 1 対 1 に対応付けて比較する。
i 番目のステップの一致度 \(s_i\) は次のように定義する。

\[
s_i =
\begin{cases}
1.0, & \text{if } \sigma^{(1)}_i = \sigma^{(2)}_i \\
0.5, & \text{if } \mathrm{type}(\sigma^{(1)}_i) = \mathrm{type}(\sigma^{(2)}_i)
        \text{ and } \mathrm{targetLabel}(\sigma^{(1)}_i) = \mathrm{targetLabel}(\sigma^{(2)}_i) \\
0,   & \text{otherwise}
\end{cases}
\]

このとき、ステップ列全体の一致度 \(S_{\mathrm{steps}}\) を

\[
S_{\mathrm{steps}} =
\frac{1}{L} \sum_{i=1}^{\min(L_1, L_2)} s_i
\]

と定義する。
\(\min(L_1, L_2)\) 個以降の「余ったステップ」は \(s_i = 0\) とみなされるため、
ステップ数の差が大きい場合にはスコアが低下する。

最後に、flow 単位のスコア \(S_{\mathrm{flow}}(f_1, f_2)\) を

\[
S_{\mathrm{flow}}(f_1, f_2)
= S_{\mathrm{op}} + 0.6 \cdot S_{\mathrm{steps}}
\]

と定義する。
これにより、operationId の一致(0.4)と steps 構造の一致(最大 0.6)を合わせて、
1件の flow ペアあたり 0.0〜1.0 の範囲でスコアを与える。
      % 評価方法
\section{実験結果}
\label{sec:experimental_results}

本節では、生成成功率、操作フローの正確性、生成の一貫性、生成時間、およびトークン効率の観点から実験結果を詳述する。

\subsection{生成成功率}
各モデルにおける生成成功率(構文エラーなく実行可能なコードが生成された割合)を表\ref{tab:success_rate}に示す。
GPT-4o は 30 回の試行中 28 回成功し、約 93.3\% という高い成功率を記録した。
一方、GPT-4o-mini の成功率は 70.0\% に留まった。
失敗の主な原因は、出力された JSON の形式不備(カンマの欠落やネスト構造の誤り)であり、
複雑なスキーマ定義を遵守する能力に関しては、ハイエンドモデルである GPT-4o が優位であることが確認された。

\begin{table}[tb]
  \centering
  \caption{モデルごとの生成成功率}
  \label{tab:success_rate}
  \begin{tabular}{l|rrr}
    \hline
    モデル & 試行回数 & 成功回数 & 成功率 \\
    \hline \hline
    GPT-4o & 30 & 28 & 93.3\% \\
    GPT-4o-mini & 30 & 21 & 70.0\% \\
    \hline
  \end{tabular}
\end{table}

\subsection{操作フローの正確性}

\begin{table}[tb]
  \centering
  \caption{操作フロー抽出の混同行列および精度評価}
  \label{tab:flow_accuracy_detail}
  \begin{tabular}{l|rrr|rrr}
    \hline
    モデル & TP & FP & FN & Precision & Recall & F1-score \\
    \hline \hline
    GPT-4o & 66 & 36 & 53 & 0.647 & 0.555 & 0.597 \\
    GPT-4o-mini & \textbf{76} & \textbf{12} & \textbf{23} & \textbf{0.864} & \textbf{0.768} & \textbf{0.813} \\
    \hline
  \end{tabular}
\end{table}

\subsubsection{操作抽出の適合率・再現率}
本評価実験において,API 生成プロセスが正常に完了したケースを分析対象とし, 
抽出された操作フロー(\texttt{steps})が正解データとどの程度一致しているかについて定量的な評価を行った。
表\ref{tab:flow_accuracy_detail}には,評価指標として採用した 
True Positive (TP),False Positive (FP),False Negative (FN) の具体的な件数内訳を示す。 
さらに,これらの値に基づいて算出された適合率(Precision),再現率(Recall),
および総合的な精度を示す F1 スコアについても併せて掲載する。

表\ref{tab:flow_accuracy_detail}の結果から明らかであるように,
本タスクにおいては GPT-4o-mini が GPT-4o を大きく上回る性能を発揮し,
F1 スコアにおいて 0.813 という高い値を記録した。 
モデルごとの具体的な内訳を詳細に見ると,以下の顕著な特徴が確認された。

\begin{itemize}
  \item \textbf{GPT-4o の課題}: FN(抽出漏れ)が 53 件と非常に多い。これは、マニュアルに記載されている操作の約半数を見落としていることを意味する。また、FP(過剰抽出)も 36 件あり、存在しない操作を誤って生成する傾向も見られた。
  \item \textbf{GPT-4o-mini の特性}: FP が 12 件、FN が 23 件といずれも低い値に抑えられている。これは、マニュアルの記述に対して忠実であり、かつ必要な操作を網羅的に抽出できていることを示している。
\end{itemize}

一般にパラメータ数が多いモデルの方が性能が高いとされるが、特定のフォーマットに従って情報を抽出する本タスクにおいては、軽量モデルである GPT-4o-mini の方が高い抽出能力を示した。

\subsubsection{ステップ順序の正確性}
抽出された操作(TP)において、ボタンを押す順序などのステップ構成が正しいかを検証した。
結果として,両モデルともにステップ順序の正確性は極めて高く,
GPT-4o で 0.94,GPT-4o-mini で 0.95 という高水準な値を記録した。
この結果は,ひとたび操作の存在自体を正しく認識できさえすれば,
その後の具体的な手順(ステップ)の因果関係や前後関係については,
どちらのモデルもマニュアルの記述通りに正確に再現・生成できる能力を有していることを示唆している。

\subsection{生成の一貫性}
同一のマニュアルを複数回入力した際の出力の安定性を評価するため,フロー一致率を測定した結果を表\ref{tab:consistency}に示す。
実験の結果,GPT-4o-mini の平均一致率は 0.609 となり,GPT-4o (0.462) よりも有意に高い一貫性を示した。
一般に GPT-4o は高い表現力を持つとされるが,本タスクにおいてはその特性が裏目に出ている可能性がある。
すなわち,試行ごとに操作名の命名規則(例: \texttt{turnOn} と \texttt{powerOn} の揺らぎ)や手順の粒度解釈が変動しやすく,
これが構造化データの生成における不安定要因となっていると考えられる。

\begin{table}[t]
  \centering
  % --- 左側の表: 一貫性 ---
  \begin{minipage}[t]{0.48\linewidth}
    \centering
    \caption{生成結果の一貫性(フロー一致率)}
    \label{tab:consistency}
    \begin{tabular}{l|r}
      \hline
      モデル & 平均一致率 \\
      \hline \hline
      GPT-4o & 0.462 \\
      GPT-4o-mini & \textbf{0.609} \\
      \hline
    \end{tabular}
  \end{minipage}
  \hfill % ← ここで左右の間隔を調整
  % --- 右側の表: 生成時間 ---
  \begin{minipage}[t]{0.48\linewidth}
    \centering
    \caption{平均生成時間}
    \label{tab:generation_time}
    \begin{tabular}{l|r}
      \hline
      モデル & 平均時間 (秒) \\
      \hline \hline
      GPT-4o & \textbf{23.57} \\
      GPT-4o-mini & 31.10 \\
      \hline
    \end{tabular}
  \end{minipage}
\end{table}
\subsection{生成時間}
表\ref{tab:generation_time}に平均生成時間を示す。
本実験環境においては、GPT-4o が約 23.6 秒、GPT-4o-mini が約 31.1 秒となり、
パラメータ数の多い GPT-4o の方がむしろ高速に処理を完了するという結果が得られた。
後述するトークン数の内訳に基づくと、GPT-4o-mini は出力トークン数が非常に多いため、
生成にかかるレイテンシが増大したと考えられる。

\subsection{トークン効率}
1回のドライバ生成プロセスにおける平均消費トークン数の内訳(入力プロンプト、出力完了、合計)を表\ref{tab:token_breakdown}に示す。
一般に、軽量モデルである GPT-4o-mini はコスト効率に優れるとされるが、本研究においては、GPT-4o の方がより少ないトークン数で処理を完了できるという結果が得られた。

\begin{table}[tb]
  \centering
  \caption{平均消費トークン数の内訳 (Tokens/run)}
  \label{tab:token_breakdown}
  \begin{tabular}{l|rrr}
    \hline
    モデル & Input Tokens & Output Tokens & Total Tokens \\
    \hline \hline
    GPT-4o & \textbf{33,729} & \textbf{702} & \textbf{34,431} \\
    GPT-4o-mini & 40,160 & 860 & 41,020 \\
    \hline
  \end{tabular}
\end{table}

表\ref{tab:token_breakdown} の内訳に基づき、以下の特性が確認された。

\begin{itemize}
  \item \textbf{Input Tokens}: 
  同一のマニュアル(PDF)を入力しているにもかかわらず、GPT-4o (33,729) は GPT-4o-mini (40,160) と比較して約 16\% 少ないトークン数で入力を処理している。
  これは、GPT-4o のトークナイザが、日本語の技術文書や PDF に含まれる特殊文字に対して、より高い圧縮効率を持っていることを示唆している。

  \item \textbf{Output Tokens}:
  出力においても、GPT-4o (702) は GPT-4o-mini (860) より約 18\% 少ないトークン数で生成を完了している。
  両モデルともに構文的に正しい JSON を生成しようとするが、GPT-4o の方がより簡潔で無駄のない記述を行う能力に長けていると考えられる。
\end{itemize}

以上の結果から、単価(Cost per Token)においては GPT-4o-mini が安価であるが、処理するトークン総量(Volume)においては GPT-4o が効率的であることが明らかになった。
システム全体のレイテンシや、コンテキストウィンドウの消費を抑える観点では、ハイエンドモデルである GPT-4o の利用が有利に働くケースが存在するといえる。
      % 結果
      % 第6章 評価実験
\chapter{考察および今後の課題}

\section{有効性のまとめ}
自動生成手法の有用性についてまとめる。

\section{課題と限界}
現時点での制約や失敗要因を整理する。

\section{今後の展望}
本研究の発展方向を示す。
      % 第7章 考察および今後の課題
\chapter{結論}

\section{まとめ}
本研究で得られた成果と提案手法の有効性を総括する。

\section{今後の展望}
今後の研究・開発の方向性を述べる。
      % 第8章 結論
\appendix

\backmatter% ここから後付
\chapter{謝辞}
本研究の遂行にあたり,終始懇切丁寧なご指導,ご鞭撻を賜りました,指導教員である東洋大学情報連携学部 矢代 武嗣 教授に深く感謝の意を表します。先生の熱心なご指導のおかげで,本研究を形にすることができました。

また,共に研究活動に励み,有意義な時間を共有させていただいた矢代研究室の諸氏に厚く御礼申し上げます。

最後に,在学中の生活を支え,常に温かく見守ってくれた家族,および友人に心からの感謝を捧げます。
             % 謝辞

\bibliography{thesis.bib}  % 参考文献

\appendix% ここから付録 %%%%% 付録 %%%%%%%
\chapter*{付録}
\addcontentsline{toc}{chapter}{付録}
\label{chap:appendix}

% ==========================================
% A. CLIツールの使い方
% ==========================================
\section*{A. CLI コマンドリファレンス}
本研究で開発した CLI ツール \texttt{auto-iot} の主要コマンドおよびその仕様を以下に示す。

\subsection*{A.1 init-bot-map}
SwitchBot API と通信を行い、アカウントに紐付いている物理デバイスの一覧を取得して、マッピング定義ファイルの雛形 (\texttt{switchbot.map.json}) を生成する。

\textbf{使用法:}
\begin{verbatim}
$ auto-iot init-bot-map [options]
\end{verbatim}

\textbf{オプション:}
\begin{itemize}
  \item \texttt{-d, --output-dir <dir>}: 出力先のディレクトリを指定する。ファイル名は \texttt{switchbot.map.json} で固定される (デフォルト: \texttt{.} )。
  \item \texttt{-c, --conflict <mode>}: 同名のファイルが既に存在する場合の挙動を指定する。
  \begin{itemize}
    \item \texttt{overwrite}: 上書きする (デフォルト)。
    \item \texttt{add}: 既存のファイルに新しいデバイスを追記する。
    \item \texttt{skip}: 生成をスキップする。
  \end{itemize}
  \item \texttt{-h, --help}: ヘルプを表示する。
\end{itemize}

\subsection*{A.2 generate-doc (Analyzer)}
マニュアル (PDF/テキスト) を解析し、中間表現 (IR) および OpenAPI 仕様書を生成する。
本コマンドの実行には \texttt{OPENAI\_API\_KEY} が環境変数として必要である。

\textbf{使用法:}
\begin{verbatim}
$ auto-iot generate-doc [options]
\end{verbatim}

\textbf{オプション:}
\begin{itemize}
  \item \texttt{--doc <file>}: 解析対象のマニュアルファイルのパス (PDF またはテキスト)。
  \item \texttt{--map <file>}: \texttt{init-bot-map} で作成したデバイスマップのパス。
  \item \texttt{--base <file>}: 既存の API 仕様書がある場合、それをベースとして読み込む (省略可)。
  \item \texttt{-o, --output-dir <dir>}: 生成物 (\texttt{openapi.json}, \texttt{openapi.yaml}, \texttt{switchbot\_flow.json}) を保存するディレクトリ。
  \item \texttt{-h, --help}: display help for command
\end{itemize}

\subsection*{A.3 implement}
% ここに implement コマンドの使用例と、ロジック注入の挙動について記述する

% ==========================================
% B. LLMへのプロンプト
% ==========================================
\section*{B. プロンプト定義}
マニュアル解析フェーズにおいて LLM (GPT-4o) に与えた指示の構成を以下に示す。

\subsection*{B.1 System Prompt}
% ここに System Prompt の実物(または主要な制約部分)を枠付きで掲載する

% ==========================================
% C. 生成されるデータの例
% ==========================================
\section*{C. 生成データの詳細例}
システムが出力する中間ファイルおよび最終成果物の構造例を示す。

\subsection*{C.1 中間表現 (IR) の生成例}
% ここに ir.json の実例(JSONデータ)を掲載する

\subsection*{C.2 注入後のサーバコード例}
% ここに default_controller.py の実例(Pythonコード、注入前後がわかるもの)を掲載する

% ==========================================
% D. 主要なソースコード
% ==========================================
\section*{D. システムの主要実装コード}
本システムの動作原理を支える重要なコンポーネントの実装を抜粋して示す。

\subsection*{D.1 SwitchBot ドライバ (\texttt{switchbot\_driver.py})}
% ここに Python ドライバのコード(署名生成や通信部分)を掲載する

\subsection*{D.2 ロジック注入処理 (TypeScript 抜粋)}
% ここに implement.ts 内の「正規表現でマーカーを探して置換するロジック」を掲載する
 % 付録

\end{document}
